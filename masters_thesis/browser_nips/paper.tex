\documentclass{article}
\usepackage{nips10submit_e,times}
\usepackage{graphicx}

\title{The Topic Browser\\An Interactive Tool for Browsing Topic Models}
\author{
Matthew J. Gardner \\
Department of Computer Science \\
Brigham Young University \\
\texttt{mjg82@byu.edu} \\
\And
Joshua Lutes \\
Department of Computer Science \\
Brigham Young University \\
\texttt{joshua@byu.edu} \\
\And
Jeff Lund \\
Department of Computer Science \\
Brigham Young University \\
\texttt{jefflund@gmail.com} \\
\And
Josh Hansen \\
Department of Computer Science \\
Brigham Young University \\
\texttt{jjfresh@byu.edu} \\
\And
Dan Walker \\
Department of Computer Science \\
Brigham Young University \\
\texttt{danwalkeriv@gmail.com} \\
\And
Eric Ringger \\
Department of Computer Science \\
Brigham Young University \\
\texttt{ringger@cs.byu.edu} \\
\And
Kevin Seppi \\
Department of Computer Science \\
Brigham Young University \\
\texttt{kseppi@cs.byu.edu} \\
}


\begin{document}
\nipsfinalcopy % Uncomment for camera-ready version
\maketitle

\begin{abstract}

Topic models have been shown to reveal the semantic content in large corpora.
Many individualized visualizations of topic models have been reported in the
literature, showing the potential of topic models to give valuable insight into
a corpus.  However, good, general tools for browsing the entire output of a
topic model along with the analyzed corpus have been lacking.  We present an
interactive tool that incorporates both prior work in displaying topic models
as well as some novel ideas that greatly enhance the visualization of these
models.

\end{abstract}

\section{Introduction}

Proper visualizations are essential to extracting information from and
identifying trends in data, especially large, high dimensional data.  Large
text corpora are particularly difficult to visualize, as they may include many
thousands of documents and millions of words.  Topic
modeling~\cite{blei-2003-latent-dirichlet-allocation} is a method of reducing
the dimensionality of a corpus into a set of meaningful topics.  Topic models
have been used with some success as an aid to corpus
visualization~\cite{blei-2009-topic-models,
newman-2010-visualizing-with-topic-maps}.  However, in many papers the
presentation of topic models has been limited to hand-selected, coherent
topics, when often there are many meaningless topics to sift through.  These
presentations leave the viewer wondering what else the topic model discovered
and provide little help in gaining a deep understanding of the model.

To aid in the visualization of topic models and in pattern discovery in
document collections, we present The Topic Browser, a web-based tool for
interactively exploring both the output of a topic modeling algorithm and its
attendant corpus.  Our topic browser incorporates many visualizations of topic
models previously published as well as some innovative ideas of our own.  The
Topic Browser is an aid both to those who wish to browse through a corpus and
for those who wish to analyze the topic model itself.  We believe that this
interactive approach to visualization in browsing reveals topics and topical
trends in large document collections in a way that is not possible with static
visualizations.  The rest of this paper describes the Topic Browser in an
attempt to demonstrate that this is true.  We display our tool with a topic
model run using MALLET's implementation of LDA~\cite{mallet} with 150 topics on
a collection of about 460 campaign speeches from the 2008 presidential primary
and general elections retrieved from http://2008election.procon.org.

\section{Extra Information}

Aside from the documents and the topic model themselves, our browser
incorporates three other pieces of information: attributes (metadata)
associated with each document, topic metrics, and document metrics.  Here we
give a brief description of these kinds of information, deferring a discussion
of their use in visualization to the subsequent sections.

Attributes of documents have been used heavily in recent topic models,
including the Author-Topic model~\cite{rosen-zvi-2004-author-topic-model},
Topics over Time~\cite{wang-2006-topics-over-time}, and Dirichlet-Multinomial
Regression~\cite{mimno-2008-topic-models-with-arbitrary-features-dmr}.  While
we currently do not have specialized visualizations for these topic models, we
do include document attributes in our browser and have some of our own 
visualizations that include the attributes.

In order to browse more effectively through topics, we introduce topic metrics
that give information about the topic.  These range from simple metrics, such
as the number of word tokens and types labeled with the topic, to more
complicated metrics such as how dispersed the topic is across documents, or how
coherent its words
are~\cite{newman-2010-automatic-evaluation-of-topic-coherence}.  We also use
pairwise topic metrics as similarity measures to show similar topics.

Similar to topic metrics, one can also compute document metrics.  Beyond simple
metrics like token count in the document, these include things such as the
entropy of the topic distribution of the
document~\cite{misra-2008-lda-to-find-semantically-incoherent-documents}.  And
as with topics, we make use of pairwise document metrics such as topic
correlation~\cite{blei-2009-topic-models} to show similar documents.

\section{Sidebar}

The sidebar in our browser is the main navigation tool.  To facilitate
navigation through the large amount of information in the browser, the sidebar
lists the items of the type currently being explored.  The new capability
introduced by this browser is the ability to sort and filter these lists as the
user desires.

When presenting the results of a topic model analysis, papers will often hand
pick particularly good topics to display, leaving out a large number of
meaningless topics.  While such presentations may help to highlight the
benefits of a new topic model, they hide the fact that finding good topics is
often a laborious process.  With our topic and document metrics, the sorting
and filtering capabilities of the sidebar list allow the user to go quickly to
meaningful topics that are relevant to his questions.

When browsing through topics, the user can filter the topic list by coherence
to eliminate from the view those topics that are mostly meaningless and sort by
document entropy (a measure of the dispersion of the topic across the
documents) to find topics that were used widely throughout the corpus.  For
example, Figure~\ref{fig:sidebar} shows that ``the family'' was one of the most
consistent themes in the speeches.  We use the top two words to name a topic in
the sidebar and other places, to avoid clutter, though we are actively
investigating better ways to name topics.

\begin{figure}
  \centering
  \includegraphics[width=.34\textwidth]{sidebar}
  \caption{The navigational sidebar showing topics.  In this example the topics
  are sorted by dispersion throughout the corpus and filtered by coherence}
  \label{fig:sidebar}
\end{figure}

The user can also filter by document or attribute to show only topics that were
used in a particular document or by a particular author.  Documents can
similarly be filtered and sorted, showing only documents that contain tokens
with a particular topic or with a particular value of an attribute.

\section{Topic Browsing}

The results from a topic model are typically presented as static lists of
words, often simply showing the top ten words from each topic in a
list~\cite{blei-2003-latent-dirichlet-allocation}.  While this practice is
often sufficient to give a reader a basic idea of what the topic captured, it
is largely incapable of conveying a deep understanding of the context of that
topic in the corpus.  In order to answer significant questions about a corpus
one must browse the topics discovered, and previous methods have failed to
adequately allow the user to explore topics.  Our browser presents new ways to
gather information about a topic in the corpus, both in terms of how the words
in the topic are displayed and in terms of what is presented about the topic.

\subsection{Showing Top Words}

Instead of showing a list of the top ten words, we show a word cloud of the top
100, where the size of each word is determined by the probability of seeing
that word in the topic.  We also show a re-weighted word cloud as determined by
Blei \& Lafferty's Turbo Topics method~\cite{blei-2009-turbo-topics}.

While these visualizations are modest improvements over typical visualization
methods, our most useful display of the words is showing them in their context.
The topics in a topic model cannot be fully interpreted when completely
separated from their context.  Thus in addition to the word cloud we also
display the top ten words in the topic inside of their context.  We select a
random token of each word type labeled with the topic and show up to 50
characters on either side of the token, keeping words intact.  We also allow
the user to cycle through contexts to gain a broader view of how each word is
actually used in that topic.  When viewing the topic about troops in Iraq, for
example (see Figure~\ref{fig:context}), one can see that most often when the
word ``troops'' is used in this topic, it is in reference to bringing the
troops home, and when ``end'' is used, it is not used in the context of ``end
the war'' as often as one might expect.

\begin{figure}
  \centering
  \includegraphics[width=.8\textwidth]{words_in_context}
  \caption{The top ten words in the topic shown with a random context selected
  from the corpus.}
  \label{fig:context}
\end{figure}

\subsection{Getting More Information}

The browser provides two main ways of getting more information about topics in
the model.  The first is showing documents and attribute values which have
either the highest number of tokens labeled with that topic or the highest
proportion (by percentage) of the given topic.  This allows the user to quickly
find documents that best demonstrate what the topic captured.  Showing the top
values for a given attribute (such as authors for the attribute ``Author'')
gives the user an idea of how focused the topic is and often reveals
interesting information about the corpus being browsed.  For example, in Figure
\ref{fig:top-values}, we see that Democrats spoke more often about education
than Republicans did, at least in the topic shown.

\begin{figure}
  \centering
  \includegraphics[width=\textwidth]{top_values}
  \caption{Part of the overall topic view, looking at a topic about education.
  Note particularly the top documents for the topic and the top values for the
  attribute ``party,'' shown at the bottom.}
  \label{fig:top-values}
\end{figure}

The other way that we give more information about topics is by showing similar
topics.  We find similar topics either by looking at the distribution of
documents that contain the topic or by the topic's distribution over words.
Looking at similar topics by document shows topics that are commonly used
together in the corpus, and finding similar topics by word distribution shows
topics that have similar words, possibly because the two topics really should
have been one topic.  When looking at a topic about health care, the user can
see that other topics used together with the health care topic include topics
about the cost of medication and the quality of care (see
Figure~\ref{fig:similar}).

\begin{figure}
  \centering
  \includegraphics[width=\textwidth]{similar_topics}
  \caption{Another look at the overall topic view, featuring different parts of
  the view.  Note particularly the similar topics box near the top right.}
  \label{fig:similar}
\end{figure}

\section{Document, Word, and Attribute Browsing}

In addition to enabling the user to browse the topics in the topic model, we
provide means for browsing the documents, words, and attributes in the corpus.
These facilities give users the ability to explore the corpus itself in the
context of the topic model.

When simply browsing through the documents, we provide sorting and filtering
methods on the list of documents, as mentioned previously.  When looking at a
particular document, we show basic information about the document, its text,
the topic distribution in the document, and similar documents based on that
distribution.  When looking at the document in the context of a topic, we also
highlight the tokens in the document that were labeled with that topic.  For
example, the user might be curious to see a document that best demonstrates the
health care topic mentioned above.  Clicking on the top document in the
document list (as in the bottom of Figure~\ref{fig:top-values}) brings the user
to the view shown in Figure~\ref{fig:document}.

\begin{figure}
  \centering
  \includegraphics[width=\textwidth]{document}
  \caption{The document view, looking at a speech by Hillary Clinton in the
  context of a topic about health care (the rest of the document is cut off in
  this screenshot, and sadly, the colored tokens in the document are not very
  visible in black and white).}
  \label{fig:document}
\end{figure}

The document visualizations reported here were drawn from the work of
others~\cite{blei-2003-latent-dirichlet-allocation}, and in fact the view in
Figure~\ref{fig:document} constitutes the entirety of most previous corpus
browsers based on topic models (e.g.,~\cite{blei-arxiv-corpus-browser}).  Our
browser substantially goes beyond the existing functionality reported by
others.

We also provide views of individual words in the corpus.  When viewing a word
in the context of a topic, the user can see all uses of the word in that topic
in the corpus, with context taken from their corresponding documents.  The user
can also view words independently with a search-like interface, seeing topics
and documents in which the word appears most frequently.  The user examining
the health care topic might be curious where else the word ``health'' was used
in the topics and in the corpus.  Figure~\ref{fig:word} shows the result of
using our word search to answer that question.  While providing basic
functionality, however, the search interface leaves much to be desired, as only
single words can currently be searched for.  We plan on expanding that to
phrases.

\begin{figure}
  \centering
  \includegraphics[width=.72\textwidth]{word}
  \caption{The word search functionality, showing the results for searching for
  the word ``health.''}
  \label{fig:word}
\end{figure}

The user can also look at aggregated information for the values of an attribute
(i.e., a particular candidate or party), combining all of the topic and word
counts for all documents with the given attribute.  This view is also at
present somewhat limited, showing only what topics and words are used most
frequently by the collection of documents with that attribute.
Figure~\ref{fig:attribute} shows us, among other things, that one of Barack
Obama's top topics was ``change in politics.''

\begin{figure}
  \centering
  \includegraphics[width=\textwidth]{attribute}
  \caption{The attribute view, showing aggregated information for all speeches
  given by Barack Obama.}
  \label{fig:attribute}
\end{figure}

\section{Plots}

We currently include two kinds of plots in our topic browser, with plans to
implement many more.  The first shows trends for topics over the values of an
attribute (such as date, or candidate), useful for corpus browsing.  This kind
of plot has been used in visualizing topic models almost since their
introduction~\cite{griffiths-2004-finding-scientific-topics}.  Our topic
browser allows the user to interactively generate these trend plots over any
attribute for any topic or combination of topics in the corpus.  Our user
exploring health care topics may want to view how much each candidate spoke
about health care.  We saw already that there were three related topics that
mentioned health care, so the user might view all of them together in a
histogram, shown in Figure~\ref{fig:trend-plot}.

\begin{figure}
  \centering
  \includegraphics[width=\textwidth]{trend_plot}
  \caption{A plot of topics over attributes, showing the use of three health
  care-related topics across candidates.}
  \label{fig:trend-plot}
\end{figure}

The second kind of plot that we include is more useful for analyzing and
understanding the behavior of the topic model itself.  We allow the user to
plot two topic metrics against each other and compute a linear regression.
This allows the user to see some interesting properties of the topic metrics,
such as the fact that document entropy seems to correlate with the logarithm of
the number of tokens in the topic, and that coherence does not seem to
correlate with any other topic metric.  The user can also find outliers, such
as topics with low document entropy but a high token count, that can then be
examined in the topic page.

An interesting application of these topic metric plots occurs when the metrics
include how consistently each candidate spoke about each topic.
Figure~\ref{fig:metric-plot} shows a plot of topics, comparing John McCain's
use of each topic to Barack Obama's use of the topic.  Topics in the upper left
were topics unique to Obama, and topics in the lower right were unique to
McCain.  Topics in the middle (near the value of 2 for each candidate) were
somewhat shared between the two.

\begin{figure}
  \centering
  \includegraphics[width=\textwidth]{metric_plot}
  \caption{A topic metric comparison plot.  The metrics plotted are how
  consistently Barack Obama and John McCain used each topic. The plot shows
  topics that were unique to Obama or McCain and topics they shared.}
  \label{fig:metric-plot}
\end{figure}

\section{Conclusion}

We have presented the Topic Browser, an interactive tool for browsing both the
output of a topic model and the corpus that was modeled.  We have shown that
our tool incorporates many previously published visualizations of topic models,
including basic corpus browsing functionality, plots of trends over attributes
in the corpus, and Blei \& Lafferty's Turbo Topics method of finding
significant phrases for each topic.  We have also presented many novel ways to
mine information from a topical analysis of a corpus in an interactive browsing
experience.  The Topic Browser is an effective tool both for those wishing to
browse through a corpus in the context of a topic model, and for those wishing
to better understand topic models and develop new models.


While our tool is still under development, a description of the Topic Browser,
a working demo, and the current version of the code are available at
http://nlp.cs.byu.edu/topic\_browser [Note to reviewers: this is currently just
a demo of the browser; by the time of the workshop it will have the description
and code mentioned].  Our tool currently supports any topic model that labels
individual tokens in the corpus with topics, and is built to import data
directly from MALLET input and output files~\cite{mallet}, or files similarly
formatted.  We have plans to include specialized visualizations for more
complicated topic models, such as Topics over Time, sentiment-topic models,
hierarchical topic models, and others.

\bibliographystyle{plain}
\bibliography{../../../bib/lda/bib}

\end{document}
