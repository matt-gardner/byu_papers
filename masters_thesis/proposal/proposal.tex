\documentclass[ms]{byuprop}
% Options for this class include the following (* indicates default):
%
%   10pt -- 10 point font size
%   11pt -- 11 point font size
%   12pt (*) -- 12 point font size
%
%   ms -- produce a thesis proposal (off)
%   areaexam -- produce a research area overview (off)
%   phd -- produce a dissertation proposal (off)
%   
%   layout -- show layout lines on the pages, helps with overfull boxes (off)
%   grid -- show a half-inch grid on every page, helps with printing (off)


% This command fixes my particular printer, which starts 0.03 inches too low,
% shifting the whole page down by that amount.  This shifts the document
% content up so that it comes out right when printed.
%
% Discovering this sort of behavior is best done by specifying the ``grid''
% option in the class parameters above.  It prints a 1/2 inch grid on every
% page.  You can then use a ruler to determine exactly what the printer is
% doing.
%
% Uncomment to shift content up (accounting for printer problems)
%\setlength{\voffset}{-.03in}

% Here we set things up for invisible hyperlinks in the document.  This makes
% the electronic version clickable without changing the way that the document
% prints.  It's useful, but optional.
\usepackage[
    ps2pdf,
    bookmarks=true,
    breaklinks=true,
    raiselinks=true,
    pdfborder={0 0 0},
    colorlinks=false,
    ]{hyperref}

% Rewrite the itemize, description, and enumerate environments to have more
% reasonable spacing:
\newcommand{\ItemSep}{\itemsep 0pt}
\let\oldenum=\enumerate
\renewcommand{\enumerate}{\oldenum \ItemSep}
\let\olditem=\itemize
\renewcommand{\itemize}{\olditem \ItemSep}
\let\olddesc=\description
\renewcommand{\description}{\olddesc \ItemSep}

% Get a little less fussy about word spacing on a line.  Sometimes produces
% ugly results, so keep your eyes peeled.
\sloppy

% Important settings for the byuprop class. %
%%%%%%%%%%%%%%%%%%%%%%%%%%%%%%%%%%%%%%%%%%%%%

% Because I use these things in more than one place, I created new commands for
% them.  I did not use \providecommand because I absolutely want LaTeX to error
% out if these already exist.
\newcommand{\Title}{Improved Corpus Exploration with Topic Modeling}
\newcommand{\Author}{Matthew Gardner}
\newcommand{\SubmissionMonth}{January}
\newcommand{\SubmissionYear}{2011}

% Take these from the commands defined above
\title{\Title}
\author{\Author}
\monthsubmitted{\SubmissionMonth}
\yearsubmitted{\SubmissionYear}

% Committee members
\committeechair{Kevin Seppi}
\committeemembera{Eric Ringger}
\committeememberb{David Embley}
%\committeememberc{}
%\committeememberd{}

% Department graduate coordinator
\graduatecoordinator{Kent Seamons}

%%%%%%%%%%%%%%%%%%%%%%%%%%%%%%%%%%%%%%%%%%%%%

% Set up the internal PDF information so that it becomes part of the document
% metadata.  The pdfinfo command will display this. Be sure to set the document
% type and add your own keywords.
\hypersetup{%
    pdftitle=\Title,%
    pdfauthor=\Author,%
    pdfsubject={Masters thesis proposal, BYU CS Department: %
                Submitted \SubmissionMonth~\SubmissionYear, Created \today},%
    pdfkeywords={topic modeling, corpus browsing, corpus discovery, %
				text mining},%
}

% These packages allow the bibliography to be sorted alphabetically and allow references to more than one paper to be sorted and compressed (i.e. instead of [5,2,4,6] you get [2,4-6])
\usepackage{url}
\usepackage[numbers,sort&compress]{natbib}
\bibliographystyle{annotnat}
\usepackage{hypernat}
% Note, use \citet to name the authors (text mode) and \citep to do just give
% the number (parenthetical mode).  Also, \citep* will give all of the authors
% rather than an abbreviated list.

% Additional packages required for your specific thesis go here. I've left some I use as examples.
\usepackage{graphicx}
%\usepackage{pdfsync}
\usepackage{amsmath}


\begin{document}

% Produce the preamble
\maketitle


%%%%%%%%%%%%%%%%%%%%%%%%%%%%%%%%%%%%%%%%%%%%%%%%%%%%%%%%%%%%%%%%%%%%%%%%%%%%%%

\section{Abstract}
% 1 to 2 paragraphs


%%%%%%%%%%%%%%%%%%%%%%%%%%%%%%%%%%%%%%%%%%%%%%%%%%%%%%%%%%%%%%%%%%%%%%%%%%%%%%

\section{Introduction}
% 1 to 4 pages


%%%%%%%%%%%%%%%%%%%%%%%%%%%%%%%%%%%%%%%%%%%%%%%%%%%%%%%%%%%%%%%%%%%%%%%%%%%%%%

\section{Related Work}
% 1 to 2 pages


%%%%%%%%%%%%%%%%%%%%%%%%%%%%%%%%%%%%%%%%%%%%%%%%%%%%%%%%%%%%%%%%%%%%%%%%%%%%%%

\section{Thesis Statement}
% 1 to 2 sentences


%%%%%%%%%%%%%%%%%%%%%%%%%%%%%%%%%%%%%%%%%%%%%%%%%%%%%%%%%%%%%%%%%%%%%%%%%%%%%%

\section{Project Description}
% 2 to 5 pages


%%%%%%%%%%%%%%%%%%%%%%%%%%%%%%%%%%%%%%%%%%%%%%%%%%%%%%%%%%%%%%%%%%%%%%%%%%%%%%

\section{Validation}
% 1/2 to 2 pages


%%%%%%%%%%%%%%%%%%%%%%%%%%%%%%%%%%%%%%%%%%%%%%%%%%%%%%%%%%%%%%%%%%%%%%%%%%%%%%

\section{Thesis Schedule}
% 1/4 to 1/2 page

\begin{itemize}

\item Completed: reformulate PSO in MapReduce.

\item February 6, 2009: evaluate PSO with large swarms with a few benchmark
functions.

\item February 20, 2009: evaluate PSO with very large swarms with subswarms
and with more benchmark functions.

\item March 9, 2009: rework individual papers into a coherent thesis and
submit draft to committee chair.

\item March 20, 2009: submit thesis to second committee member for review.

\item March 30, 2009: submit thesis to third committee member for review.

\item April 10, 2009: defend thesis.

Each of the three stages of the project can be submitted as a paper to a
conference such as GECCO or CEC.

\end{itemize}


%%%%%%%%%%%%%%%%%%%%%%%%%%%%%%%%%%%%%%%%%%%%%%%%%%%%%%%%%%%%%%%%%%%%%%%%%%%%%%

\renewcommand\bibsection{\section{Annotated Bibliography}}
%\newcommand\bibpreamble{Material to be added between heading and entries.}
% 2 to 5 pages

\bibliography{}


\end{document}
