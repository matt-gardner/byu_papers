\documentclass[onecolumn, 12pt]{article}
\usepackage{fullpage}
\usepackage{graphicx}
\usepackage{cite}
\usepackage{amsmath}

\DeclareMathOperator{\URand}{U}
\providecommand{\ppos}{\ensuremath{\Vec{x}}}
\providecommand{\pvel}{\ensuremath{\Vec{v}}}
\providecommand{\gbest}{\ensuremath{\Vec{g}}}
\providecommand{\pbest}{\ensuremath{\Vec{p}}}
\providecommand{\constriction}{\ensuremath{\chi}}
\providecommand{\coeff}{\ensuremath{\phi}}
\newcommand{\bib}[1]{../../bib/#1}

\topmargin .5in


\title{Honors Thesis Proposal\\Speculative Execution in Particle Swarm
Optimization}
\author{Matthew Gardner}
\date{\today}

\begin{document}
\maketitle

\section{Background and Significance}

Optimization problems are ubiquitous in our modern world.  Businesses such as
Google need to decide where to place ads, scientists need to fit models to
data, and airlines need to schedule their flights.  All of those problems are
optimization problems, and optimization techniques can be used to find
solutions.  Optimization algorithms have been derived from all sorts of natural
phenomena.  Some methods look at the landscape of the function as they are
searching and follow hills to the best values, as in the gradient ascent
method.  Other methods take ideas from biology or metallurgy, like genetic
algorithms and simulated annealing.  Particle swarm optimization is a recently
developed optimization technique that draws on ideas from the sociology of
flocking birds.  All of these methods, however, are fundamentally sequential in
nature---they need to be run in a specific order, and because of that, they are
typically only run on one machine.  The problems people want to solve are
getting bigger, and larger problems need to run on multiple machines in
parallel.  Parallel implementations of these algorithms have been attempted,
but they do not often perform as well as sequential versions for the amount of
computation performed.  The work I propose focuses on one of those algorithms,
particle swarm optimization, and improving its performance in a large-scale,
parallel world.

\section{Statement of Intent}

I will study the feasibility and potential performance gain of a technique
which we call speculative execution in particle swarm optimization.  By
speculative execution I mean the use of additional computational resources,
presumably extra processors running in parallel, to perform two or more
iterations of particle swarm optimization at a time, thus finding better values
faster.

\section{Procedures}
\label{sec:proc}

\subsection{Particle Swarm Optimization}

Particle swarm optimization was proposed in 1995 by James Kennedy and Russell
Eberhart.  It tries to intelligently search a multi-dimensional space by
mimicking the swarming and flocking behavior of birds and other
animals.~\cite{kennedy-icnn95} It is a sociological algorithm that depends on
interaction between particles to quickly and consistently find the optimal
solution to a problem.  The algorithm keeps track of a number of potential
solutions, called particles, which move somewhat randomly through the search
space.  Particles remember the best place they have been, or solution they have
evaluated, and are attracted back to that place, as well as to the best
solution other particles have seen.  Specifically, the formulas for updating
the position $\ppos_t$ and velocity $\pvel_t$ of a particle at iteration $t$ are
as follows:
\begin{align}
\label{eq:velupdate}
	\pvel_{t+1} &=
		\constriction \left[ \pvel_t +
			\coeff_1\URand()\otimes(\pbest - \ppos_t) +
			\coeff_2\URand()\otimes(\gbest - \ppos_t)
		\right] \\
\label{eq:posupdate}
	\ppos_{t+1} &= \ppos_t + \pvel_{t+1}
\end{align}
where \( \URand() \) is a vector of random numbers drawn from a uniform
distribution, the \( \otimes \) operator is an element-wise vector
multiplication, $\pbest$ (called pbest) is the best position the current
particle has seen, and $\gbest$ (called gbest) is the best position any of the
other particles have seen.  \( \coeff_1 \), \( \coeff_2 \), and \(
\constriction \) are parameters with prescribed values required to ensure
convergence (2.05, 2.05, and .73,
respectively).~\cite{clerc-tec02}~\cite{poli-aea08}

The sociology in this algorithm defines which other particles form the
``neighborhood'' of a given particle---the other particles whose best solution
it sees.  There are many ways to define a particle's neighborhood, varying from
the entire rest of the swarm to just one other particle.  

The way a neighborhood is defined can have drastic effects on the performance
of the algorithm; the more neighbors each particle has, the faster information
spreads throughout the particles, and the quicker the algorithm converges.  For
some problems it is possible that the algorithm converges too quickly and gets
caught in a local optimum; it stops on a hill when there is a mountain next to
it.  For those kinds of problems, less communication, or smaller neighborhoods,
is often preferable.

\subsection{Speculative Execution}
\label{sec:specex}

Speculative execution in a program is the execution of code that may or may not
end up actually being needed.  Modern processors routinely do this when a
conditional branch is encountered; they try to predict which branch will
actually be needed and continue their execution.  If it turns out that the
prediction was incorrect, the work is discarded.  But if the prediction was
right, the program can execute much faster than if it had waited on the branch.

In many optimization algorithms speculative execution is impossible, because
the next iteration of the algorithm depends on the evaluation of the function
during the previous iteration.  In simulated annealing, for instance, the
probability of accepting a new position depends on the difference in function
value between the new position and the current one.  In gradient descent, the
gradient of the function must be determined (which includes actually evaluating
it) to know how far to move before sampling the function again.  But particle
swarm optimization is rather unique in that it does not require the evaluation
of the function to know what the next position of each particle will be.  It
simply depends on \emph{which} particle ended up having the best position and
whether or not the current particle found a better spot than it had seen
before.  That means you can just compute all possible next positions---all
combinations of which particle is the new gbest and whether or not pbest was
updated (\gbest\ and \pbest\ from \eqref{eq:velupdate})---and evaluate each
position in parallel.  That computes two iterations at the same time.  It uses
a lot of extra work to get the second iteration, but for some functions
performing more iterations makes more progress than adding extra particles.

\subsection{Planned Experiments}

I will implement speculative execution in particle swarm optimization.  I will
take care that my implementation can behave exactly like the original particle
swarm optimization, so a fair comparison can be made.  At each iteration a 
number of positions are speculatively evaluated, and to make it behave like
the original PSO all that needs to be done is throw away the particles that
do not match the actual execution path of the original particles.  

After showing that speculative execution reproduces PSO without affecting
results, I will expand the idea to use the information gathered from
speculative particles.  The positions that are speculatively evaluated are
potentially, and probably provably, worthwhile samples from the sampling
distribution of PSO.  Throwing them away discards useful information that could
help the algorithm in more ways than just speeding it up.  The end result would
be an algorithm that not only performs twice as many iterations as standard
PSO, but takes useful samples and potentially does better while going faster.

I will test this algorithm on standard benchmark functions, including at least
the Sphere, Rastrigin and Griewank functions.  I will compare the performance
of the speculative algorithms to the standard PSO on those functions with
varying swarm sizes.

\subsection{Evaluation}

In running experiments, I will run many trials and average the results, as
there is a random component in this algorithm.  I will use statistical tests,
such as a paired permutation test, for all comparisons to the original PSO
algorithm.  The success criterion will be a better performance of the algorithm
as measured by the lowest function evaluation seen over a number of function
evaluations.  Performing speculative execution in PSO requires more function
evaluations per particle than regular PSO, so to be fair the algorithms will be
compared with configurations where each uses the same number of function
evaluations per iteration, not the same number of particles.

\section{Preliminary Prospectus of Finished Thesis}

The finished thesis will be presented in the format of a peer-reviewed
conference paper, including the following sections:

\begin{enumerate}
  \item Introduction
  \item Related Work
  \item Particle Swarm Optimization
  \item Speculative Execution
  \item Speculative Execution in PSO
  \item Experiments
  \item Results and Conclusions
  \item Future Work
  \item References
\end{enumerate}

\section{Preliminary Research}

Before going forward with this idea, I ran some basic, preliminary tests that
show that for some functions, performing more iterations of PSO with fewer
particles does better than performing fewer iterations with more particles.
The tests were very preliminary but were enough to establish the feasibility of
using speculative execution to improve the performance of PSO on some
functions.  

Since then I have finished coding both speculative algorithms mentioned in
Section~\ref{sec:proc}.  Further, though still preliminary, testing shows again
that these methods perform better than standard PSO on a number of functions.
What remains to be done is to do more exhaustive tests, collecting results that
can be published.  I anticipate that the results that I find as I continue
testing these algorithms will drive further improvements, but the direction of
those improvements has yet to be seen.

\section{Qualifications of Investigator}

I am a Computer Science major.  My main focuses in my undergraduate career have
been optimization, natural language processing, and machine learning.  The
optimization emphasis has come mainly in my work with Dr. Kevin Seppi, for whom
I am a research assistant.  I have spent the last year working with him,
working mainly on particle swarm optimization topics.  We worked together with
Dr. Branton Campbell, a physics professor, to solve a crystallography problem
for him using a parallel implementation of PSO.  Those results will be
submitted to Nature Materials.  Recently we, along with Andrew McNabb, a
graduate student who works in the lab, submitted a paper to the Congress on
Evolutionary Computation, a major venue for optimization techniques.  The paper
was on topologies and communication in large particle swarms, and formed the
beginnings of the work I propose.

\section{Qualifications of Faculty Advisor}

Dr. Kevin Seppi is the head of the Applied Machine Learning Laboratory at BYU.
Three of his last five graduate students have done their graduate work in
particle swarm optimization, and he has authored or co-authored over 10 papers
on the topic.  I took CS 236 from him in the fall of 2007 and decided I wanted
to do research in the area he was researching, so I joined his lab.  I too
became interested in PSO and have worked in that area with him for the past
year.

\section{Schedule}

\begin{tabular}{|l|l|}
\hline
Run preliminary tests to ensure feasibility & February 28, 2009 \\
\hline
Finish writing and debugging code & April 25, 2009 \\
\hline
Finish gathering data and statistics & August 31, 2009 \\
\hline
Complete analysis of data and begin writing thesis & October 1, 2009 \\
\hline
Submit completed copy of thesis to advisor & November 15, 2009 \\
\hline
Submit paper detailing results to CEC 2010 & December 20, 2009 \\
\hline
Final thesis and portfolio submitted & February 1, 2010 \\
\hline
Thesis defense completed & March 1, 2010 \\
\hline
Final four thesis copies submitted & March 8, 2010 \\
\hline
\end{tabular}

\section{Expenses/Budget}

This project will not require any outside funding.  The only expenditure will
be my own time researching, conducting experiments, and writing.  I have
received a grant from the Office of Research and Creative Activities to perform
this research, for which I am grateful.

\section{Closure}

Thank you for reviewing my proposal.  I look forward to exploring speculative
execution in particle swarm optimization.

\bibliographystyle{plain}
\bibliography{%
\bib{pso/kennedy-icnn95},%
\bib{pso/clerc-tec02},%
\bib{pso/poli-aea08}}

\end{document}
