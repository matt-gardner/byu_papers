\documentclass[landscape,final]{baposter}

\usepackage{times}
\usepackage{calc}
\usepackage{graphicx}
\usepackage{amsmath}
\usepackage{amssymb}
\usepackage{relsize}
\usepackage{multirow}
\usepackage{bm}

\usepackage{graphicx}
\usepackage{multicol}

\usepackage{pgfbaselayers}
\pgfdeclarelayer{background}
\pgfdeclarelayer{foreground}
\pgfsetlayers{background,main,foreground}

\usepackage{helvet}
%\usepackage{bookman}
\usepackage{palatino}

\newcommand{\captionfont}{\footnotesize}

\selectcolormodel{cmyk}

\graphicspath{{images/}}

%%%%%%%%%%%%%%%%%%%%%%%%%%%%%%%%%%%%%%%%%%%%%%%%%%%%%%%%%%%%%%%%%%%%%%%%%%%%%%%%
%%%% Some math symbols used in the text
%%%%%%%%%%%%%%%%%%%%%%%%%%%%%%%%%%%%%%%%%%%%%%%%%%%%%%%%%%%%%%%%%%%%%%%%%%%%%%%%
% Format 
\newcommand{\Matrix}[1]{\begin{bmatrix} #1 \end{bmatrix}}
\newcommand{\Vector}[1]{\Matrix{#1}}
\newcommand*{\SET}[1]  {\ensuremath{\mathcal{#1}}}
\newcommand*{\MAT}[1]  {\ensuremath{\mathbf{#1}}}
\newcommand*{\VEC}[1]  {\ensuremath{\bm{#1}}}
\newcommand*{\CONST}[1]{\ensuremath{\mathit{#1}}}
\newcommand*{\norm}[1]{\mathopen\| #1 \mathclose\|}% use instead of $\|x\|$
\newcommand*{\abs}[1]{\mathopen| #1 \mathclose|}% use instead of $\|x\|$
\newcommand*{\absLR}[1]{\left| #1 \right|}% use instead of $\|x\|$

\def\norm#1{\mathopen\| #1 \mathclose\|}% use instead of $\|x\|$
\newcommand{\normLR}[1]{\left\| #1 \right\|}% use instead of $\|x\|$

%%%%%%%%%%%%%%%%%%%%%%%%%%%%%%%%%%%%%%%%%%%%%%%%%%%%%%%%%%%%%%%%%%%%%%%%%%%%%%%%
% Multicol Settings
%%%%%%%%%%%%%%%%%%%%%%%%%%%%%%%%%%%%%%%%%%%%%%%%%%%%%%%%%%%%%%%%%%%%%%%%%%%%%%%%
\setlength{\columnsep}{0.7em}
\setlength{\columnseprule}{0mm}


%%%%%%%%%%%%%%%%%%%%%%%%%%%%%%%%%%%%%%%%%%%%%%%%%%%%%%%%%%%%%%%%%%%%%%%%%%%%%%%%
% Save space in lists. Use this after the opening of the list
%%%%%%%%%%%%%%%%%%%%%%%%%%%%%%%%%%%%%%%%%%%%%%%%%%%%%%%%%%%%%%%%%%%%%%%%%%%%%%%%
\newcommand{\compresslist}{%
\setlength{\itemsep}{1pt}%
\setlength{\parskip}{0pt}%
\setlength{\parsep}{0pt}%
}


%%%%%%%%%%%%%%%%%%%%%%%%%%%%%%%%%%%%%%%%%%%%%%%%%%%%%%%%%%%%%%%%%%%%%%%%%%%%%%
%%% Begin of Document
%%%%%%%%%%%%%%%%%%%%%%%%%%%%%%%%%%%%%%%%%%%%%%%%%%%%%%%%%%%%%%%%%%%%%%%%%%%%%%

\begin{document}

%%%%%%%%%%%%%%%%%%%%%%%%%%%%%%%%%%%%%%%%%%%%%%%%%%%%%%%%%%%%%%%%%%%%%%%%%%%%%%
%%% Here starts the poster
%%%---------------------------------------------------------------------------
%%% Format it to your taste with the options
%%%%%%%%%%%%%%%%%%%%%%%%%%%%%%%%%%%%%%%%%%%%%%%%%%%%%%%%%%%%%%%%%%%%%%%%%%%%%%
\typeout{Poster Starts}
\background{
  \begin{tikzpicture}[remember picture,overlay]%
    \draw (current page.north west)+(-2em,-0em) node[anchor=north west] {\hspace{-2em}\includegraphics[height=1.1\textheight]{silhouettes_background}};
  \end{tikzpicture}%
}
\definecolor{silver}{cmyk}{0,0,0,0.3}
\definecolor{yellow}{cmyk}{0,0,0.9,0.0}
\definecolor{blue}{cmyk}{0.9,0,0,0.0}
\definecolor{reddishyellow}{cmyk}{0,0.22,1.0,0.0}
\definecolor{reddishblue}{cmyk}{1,0.22,0,0.0}
\definecolor{black}{cmyk}{0,0,0.0,1.0}
\definecolor{darkYellow}{cmyk}{0,0,1.0,0.5}
\definecolor{darkSilver}{cmyk}{0,0,0,0.1}

\definecolor{lightyellow}{cmyk}{0,0,0.3,0.0}
\definecolor{lighteryellow}{cmyk}{0,0,0.1,0.0}
\definecolor{lightestyellow}{cmyk}{0,0,0.05,0.0}
\definecolor{lightblue}{cmyk}{.3,0,0,0.0}
\definecolor{lighterblue}{cmyk}{.1,0,0,0.0}
\definecolor{lightestblue}{cmyk}{.05,0,0,0.0}
\begin{poster}{
  % Show grid to help with alignment
  grid=no,
  % Column spacing
  colspacing=1em,
  % Color style
  bgColorOne=lighterblue,
  bgColorTwo=lightestblue,
  borderColor=reddishblue,
  headerColorOne=blue,
  headerColorTwo=reddishblue,
  headerFontColor=black,
  boxColorOne=lightblue,
  boxColorTwo=lighterblue,
  % Format of textbox
  textborder=roundedleft,
  % Format of text header
  eyecatcher=no,
  headerborder=open,
  headerheight=0.08\textheight,
  headershape=roundedright,
  headershade=plain,
  headerfont=\Large\textsf, %Sans Serif
  boxshade=plain,
%  background=shade-tb,
  background=plain,
  linewidth=2pt
  }
  % Eye Catcher
  {} % No eye catcher for this poster. If an eye catcher is present, the title is centered between eye-catcher and logo.
  % Title
  {\sf %Sans Serif
  %\bf% Serif
  Automatic Topic Discovery in 100 Years of General Conference Talks}
  % Authors
  {\sf %Sans Serif
  % Serif
  Matthew Gardner
  (mjg82@byu.edu)
  and
  Eric Ringger
  (ringger@cs.byu.edu)
  }
  % University logo
  {{\begin{minipage}{16em}
    \hfill
    \includegraphics[height=5.5em]{logo}
  \end{minipage}}
  }

  \tikzstyle{light shaded}=[top color=baposterBGtwo!30!white,bottom color=baposterBGone!30!white,shading=axis,shading angle=30]

  % Width of left inset image
     \newlength{\leftimgwidth}
     \setlength{\leftimgwidth}{0.78em+8.0em}

%%%%%%%%%%%%%%%%%%%%%%%%%%%%%%%%%%%%%%%%%%%%%%%%%%%%%%%%%%%%%%%%%%%%%%%%%%%%%%
%%% Now define the boxes that make up the poster
%%%---------------------------------------------------------------------------
%%% Each box has a name and can be placed absolutely or relatively.
%%% The only inconvenience is that you can only specify a relative position 
%%% towards an already declared box. So if you have a box attached to the 
%%% bottom, one to the top and a third one which should be in between, you 
%%% have to specify the top and bottom boxes before you specify the middle 
%%% box.
%%%%%%%%%%%%%%%%%%%%%%%%%%%%%%%%%%%%%%%%%%%%%%%%%%%%%%%%%%%%%%%%%%%%%%%%%%%%%%
    %
    % A coloured circle useful as a bullet with an adjustably strong filling
    \newcommand{\colouredcircle}[1]{%
      \tikz{\useasboundingbox (-0.2em,-0.32em) rectangle(0.2em,0.32em); \draw[draw=black,fill=baposterBGone!80!black!#1!white,line width=0.03em] (0,0) circle(0.18em);}}

%%%%%%%%%%%%%%%%%%%%%%%%%%%%%%%%%%%%%%%%%%%%%%%%%%%%%%%%%%%%%%%%%%%%%%%%%%%%%%
  \headerbox{Contribution}{name=contribution,column=0,row=0}{
%%%%%%%%%%%%%%%%%%%%%%%%%%%%%%%%%%%%%%%%%%%%%%%%%%%%%%%%%%%%%%%%%%%%%%%%%%%%%%
   {}We use state of the art natural language processing techniques to gain new
   insights from 100 years of General Conference Talks.  Using computers to
   sift through a corpus that would be overwhelmingly large to process
   manually, we can discover relationships between talks and trends over time
   that help us to better appreciate the trove of knowledge in these inspired
   discourses.
 }

%%%%%%%%%%%%%%%%%%%%%%%%%%%%%%%%%%%%%%%%%%%%%%%%%%%%%%%%%%%%%%%%%%%%%%%%%%%%%%
  \headerbox{Methodology}{name=methodology,column=0,below=contribution}{
%%%%%%%%%%%%%%%%%%%%%%%%%%%%%%%%%%%%%%%%%%%%%%%%%%%%%%%%%%%%%%%%%%%%%%%%%%%%%%
	Document clustering is the task of determining which documents are similar
	to each other in any given corpus.  The fundamental assumption made by
	clustering techniques is that there are several topics used in the corpus,
	and each document is generated by a single (unknown) topic.  Given a
	clustering model and a set of documents, statistical methods can be used to
	determine the best clustering of documents for that corpus.

	When documents frequently cover multiple topics, however, the performance
	of document clustering methods degenerates.  This is often the case with
	religious discourse, as talks frequently combine faith and repentance, or
	faith and tithing, making the talks seem similar in regards to their use of
	words describing faith, but different in the other topics they cover.  To
	solve this problem, Latent Dirichlet Allocation (LDA) was proposed.  LDA
	breaks the assumption of document clustering that each document is
	generated from a single topic, and instead assumes that each word in each
	document comes from a particular topic.  Thus documents can have multiple
	topics, creating a better model for many corpora of interest today.

	LDA can be used both to discover the topics that are spoken of throughout a
	corpus and to determine which documents in the corpus use which topics most
	frequently.  When documents have dates and authors associated with them,
	other interesting attributes of the corpus, such as the relative
	proportions of topics over time, can also be deduced.  We used LDA to
	automatically discover topics found in General Conference talks from
	1900--2005.  We obtained the talks from the internet site
	www.ldslibrary.org.  There is some noise in the data because of parsing
	error, but overall the quality is high.

  }

%%%%%%%%%%%%%%%%%%%%%%%%%%%%%%%%%%%%%%%%%%%%%%%%%%%%%%%%%%%%%%%%%%%%%%%%%%%%%%
  \headerbox{Expression Neutralization}{name=results neutralization,column=1,row=0}{
%%%%%%%%%%%%%%%%%%%%%%%%%%%%%%%%%%%%%%%%%%%%%%%%%%%%%%%%%%%%%%%%%%%%%%%%%%%%%%
  \begin{tabular}{@{}c@{ }c@{ }c@{ }c@{}@{ }@{ }c@{ }c@{ }c@{ }c@{ }}
    \includegraphics[height=0.42\linewidth]{16_1_tgt}&
    \includegraphics[height=0.42\linewidth]{16_1_expression}&
    \includegraphics[height=0.42\linewidth]{16_1_neutral}\\[-0.8em]
    \smaller a) Target & \smaller b) Fit & \smaller c) Normalized\\[0.8em]
    \includegraphics[height=0.42\linewidth]{16_6_tgt}&
    \includegraphics[height=0.42\linewidth]{16_6_expression}&
    \includegraphics[height=0.42\linewidth]{16_6_neutral}\\[-0.8em]
    \smaller a) Target & \smaller b) Fit & \smaller c) Normalized
  \end{tabular}
  Expression normalisation for two scans of the same individual.  
  The robust fitting gives a good estimate (b) of the true face surface given
  the noisy measurement (a). It fills in holes and removes artifacts using
  prior knowledge from the face model. The pose and expression normalized faces
  (c) are used for face recognition.
  }
%%%%%%%%%%%%%%%%%%%%%%%%%%%%%%%%%%%%%%%%%%%%%%%%%%%%%%%%%%%%%%%%%%%%%%%%%%%%%%
  \headerbox{Robustness}{name=robustness,column=1,below=results neutralization,span=1,above=bottom}{
%%%%%%%%%%%%%%%%%%%%%%%%%%%%%%%%%%%%%%%%%%%%%%%%%%%%%%%%%%%%%%%%%%%%%%%%%%%%%%
  \begin{tabular}{@{}c@{ }c@{ }c@{ }c@{}}
    \includegraphics[height=0.42\linewidth]{56_4_tgt}&
    \includegraphics[height=0.42\linewidth]{23_2_tgt}&
    \includegraphics[height=0.42\linewidth]{5_6_tgt}\\[-0.8em]
                       & \smaller a) Targets & \\[0.8em]
    \includegraphics[height=0.42\linewidth]{56_4_expression}&
    \includegraphics[height=0.42\linewidth]{23_2_expression}& 
    \includegraphics[height=0.42\linewidth]{5_6_expression}\\[-0.8em]
                    & \smaller b) Fits & 
  \end{tabular}
  The reconstruction (b) is robust against scans (a) with artifacts, noise, and
  holes.
  }
%%%%%%%%%%%%%%%%%%%%%%%%%%%%%%%%%%%%%%%%%%%%%%%%%%%%%%%%%%%%%%%%%%%%%%%%%%%%%%
  \headerbox{Results}{name=results,column=2,span=2,row=0}{
%%%%%%%%%%%%%%%%%%%%%%%%%%%%%%%%%%%%%%%%%%%%%%%%%%%%%%%%%%%%%%%%%%%%%%%%%%%%%%
      \begin{multicols}{2}
        The method was evaluated on the GavabDB expression dataset which
        contains 427 Scans, with 3 neutral scans and 4 expression scans per ID.
        To test the impact of expression invariance on neutral data we used the
        UND Dataset from the Face Recognition Great Vendor Test, which contains
        953 neutral scans with one to eight scans per subject.
      \end{multicols}\vspace{-1em}
      \mbox{\hspace{0.3\linewidth}\rule{0.4\linewidth}{1pt}\hspace{0.3\linewidth}}\\
      \begin{tabular}{cc}
        \hspace{0.5em}\scalebox{0.74}{\input{shrec_MNCG}} &
        \hspace{0.5em}\scalebox{0.74}{\input{und_MNCG}}
      \end{tabular}\\
%      \begin{multicols}{2}
        {Expression neutralization improves results on the expression dataset
        without decreasing the accuracy on the neutral testset. Plotted is the
        ratio of correct answers to  the number of possible correct answers.
        %Note the different scales for the two graphs.
        %Our approach has a high accuracy on the neutral (UND) dataset.
        }
%      \end{multicols}\vspace{-1em}
      \\\mbox{\hspace{0.3\linewidth}\rule{0.4\linewidth}{1pt}\hspace{0.3\linewidth}}\\
      \begin{tabular}{cc}
        \hspace{0.5em}\scalebox{0.74}{\input{shrec_PR}} &
        \hspace{0.5em}\scalebox{0.74}{\input{und_PR}}
      \end{tabular}\\
%      \begin{multicols}{2}
        {Plotted are precision and recall for different retrieval depths. The lower
        precision of the UND database is due to the fact that some queries have no
        correct answers.}
%      \end{multicols}\vspace{-1em}
      \\\mbox{\hspace{0.3\linewidth}\rule{0.4\linewidth}{1pt}\hspace{0.3\linewidth}}\\
      \begin{tabular}{cc}
        \hspace{0.5em}\scalebox{0.74}{\input{shrec_FARFRR}} &
        \hspace{0.5em}\scalebox{0.74}{\input{und_FARFRR}}
      \end{tabular}\\
%      \begin{multicols}{2}
        {Impostor detection is reliable, as the minimum distance to a match
        is smaller than the minimum distance to a nonmatch. }
%      \end{multicols}
\\
  }%
%%%%%%%%%%%%%%%%%%%%%%%%%%%%%%%%%%%%%%%%%%%%%%%%%%%%%%%%%%%%%%%%%%%%%%%%%%%%%%
  \headerbox{Open Questions}{name=questions,column=2,span=1,above=bottom,below=results}{
%%%%%%%%%%%%%%%%%%%%%%%%%%%%%%%%%%%%%%%%%%%%%%%%%%%%%%%%%%%%%%%%%%%%%%%%%%%%%%
    While the expression and identity space are linearly independent, there is
    some expression left in the identity model. This is because a ``neutral''
    face is interpreted differently by the subjects. We investigate the
    possibilty to build an identity/expression separated model without using
    the data labelling, based on a measure of independence.
  }%
%%%%%%%%%%%%%%%%%%%%%%%%%%%%%%%%%%%%%%%%%%%%%%%%%%%%%%%%%%%%%%%%%%%%%%%%%%%%%%
  \headerbox{Funding}{name=funding,column=3,span=1,above=bottom}{
%%%%%%%%%%%%%%%%%%%%%%%%%%%%%%%%%%%%%%%%%%%%%%%%%%%%%%%%%%%%%%%%%%%%%%%%%%%%%%
  \smaller 
  This work was supported in part by Microsoft Research through the European PhD Scholarship Programme.
  }%
%%%%%%%%%%%%%%%%%%%%%%%%%%%%%%%%%%%%%%%%%%%%%%%%%%%%%%%%%%%%%%%%%%%%%%%%%%%%%%
  \headerbox{References}{name=references,column=3,above=funding,below=results}{
%%%%%%%%%%%%%%%%%%%%%%%%%%%%%%%%%%%%%%%%%%%%%%%%%%%%%%%%%%%%%%%%%%%%%%%%%%%%%%
    \smaller
    \vspace{-0.4em}
    \bibliographystyle{ieee}
    \renewcommand{\section}[2]{\vskip 0.05em}
      \begin{thebibliography}{1}\itemsep=-0.01em
      \setlength{\baselineskip}{0.4em}
      \bibitem{amberg07:nonrigid}
        B.~Amberg, S.~Romdhani, T. Vetter.
        \newblock {O}ptimal {S}tep {N}onrigid {ICP} {A}lgorithms for {S}urface {R}egistration
        \newblock In {\em CVPR 2007}
      \bibitem{amberg08:recognition}
        B.~Amberg, R.~Knothe, T. Vetter.
        \newblock Expression Invariant Face Recognition with a 3D Morphable Model
        \newblock In {\em AFGR 2008}
      \end{thebibliography}
  }%
\end{poster}%
%
\end{document}
