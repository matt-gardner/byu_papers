\documentclass[landscape,final]{baposter}

\usepackage{times}
\usepackage{calc}
\usepackage{graphicx}
\usepackage{amsmath}
\usepackage{amssymb}
\usepackage{relsize}
\usepackage{multirow}
\usepackage{bm}

\usepackage{graphicx}
\usepackage{multicol}

\usepackage{pgfbaselayers}
\pgfdeclarelayer{background}
\pgfdeclarelayer{foreground}
\pgfsetlayers{background,main,foreground}

\usepackage{helvet}
%\usepackage{bookman}
\usepackage{palatino}

\newcommand{\captionfont}{\footnotesize}

\selectcolormodel{cmyk}

\graphicspath{{images/}}

%%%%%%%%%%%%%%%%%%%%%%%%%%%%%%%%%%%%%%%%%%%%%%%%%%%%%%%%%%%%%%%%%%%%%%%%%%%%%%%%
%%%% Some math symbols used in the text
%%%%%%%%%%%%%%%%%%%%%%%%%%%%%%%%%%%%%%%%%%%%%%%%%%%%%%%%%%%%%%%%%%%%%%%%%%%%%%%%
% Format 
\newcommand{\Matrix}[1]{\begin{bmatrix} #1 \end{bmatrix}}
\newcommand{\Vector}[1]{\Matrix{#1}}
\newcommand*{\SET}[1]  {\ensuremath{\mathcal{#1}}}
\newcommand*{\MAT}[1]  {\ensuremath{\mathbf{#1}}}
\newcommand*{\VEC}[1]  {\ensuremath{\bm{#1}}}
\newcommand*{\CONST}[1]{\ensuremath{\mathit{#1}}}
\newcommand*{\norm}[1]{\mathopen\| #1 \mathclose\|}% use instead of $\|x\|$
\newcommand*{\abs}[1]{\mathopen| #1 \mathclose|}% use instead of $\|x\|$
\newcommand*{\absLR}[1]{\left| #1 \right|}% use instead of $\|x\|$

\def\norm#1{\mathopen\| #1 \mathclose\|}% use instead of $\|x\|$
\newcommand{\normLR}[1]{\left\| #1 \right\|}% use instead of $\|x\|$

%%%%%%%%%%%%%%%%%%%%%%%%%%%%%%%%%%%%%%%%%%%%%%%%%%%%%%%%%%%%%%%%%%%%%%%%%%%%%%%%
% Multicol Settings
%%%%%%%%%%%%%%%%%%%%%%%%%%%%%%%%%%%%%%%%%%%%%%%%%%%%%%%%%%%%%%%%%%%%%%%%%%%%%%%%
\setlength{\columnsep}{0.7em}
\setlength{\columnseprule}{0mm}


%%%%%%%%%%%%%%%%%%%%%%%%%%%%%%%%%%%%%%%%%%%%%%%%%%%%%%%%%%%%%%%%%%%%%%%%%%%%%%%%
% Save space in lists. Use this after the opening of the list
%%%%%%%%%%%%%%%%%%%%%%%%%%%%%%%%%%%%%%%%%%%%%%%%%%%%%%%%%%%%%%%%%%%%%%%%%%%%%%%%
\newcommand{\compresslist}{%
\setlength{\itemsep}{1pt}%
\setlength{\parskip}{0pt}%
\setlength{\parsep}{0pt}%
}


%%%%%%%%%%%%%%%%%%%%%%%%%%%%%%%%%%%%%%%%%%%%%%%%%%%%%%%%%%%%%%%%%%%%%%%%%%%%%%
%%% Begin of Document
%%%%%%%%%%%%%%%%%%%%%%%%%%%%%%%%%%%%%%%%%%%%%%%%%%%%%%%%%%%%%%%%%%%%%%%%%%%%%%

\begin{document}

%%%%%%%%%%%%%%%%%%%%%%%%%%%%%%%%%%%%%%%%%%%%%%%%%%%%%%%%%%%%%%%%%%%%%%%%%%%%%%
%%% Here starts the poster
%%%---------------------------------------------------------------------------
%%% Format it to your taste with the options
%%%%%%%%%%%%%%%%%%%%%%%%%%%%%%%%%%%%%%%%%%%%%%%%%%%%%%%%%%%%%%%%%%%%%%%%%%%%%%
\typeout{Poster Starts}
\definecolor{blue}{cmyk}{0.9,0,0,0.0}
\definecolor{reddishblue}{cmyk}{1,0.22,0,0.0}
\definecolor{black}{cmyk}{0,0,0.0,1.0}

\definecolor{lightblue}{cmyk}{.3,0,0,0.0}
\definecolor{lighterblue}{cmyk}{.2,0,0,0.0}
\definecolor{lightestblue}{cmyk}{.1,0,0,0.0}
\begin{poster}{
  % Show grid to help with alignment
  grid=no,
  % Column spacing
  colspacing=1em,
  % Color style
  bgColorOne=lighterblue,
  bgColorTwo=lightestblue,
  borderColor=reddishblue,
  headerColorOne=blue,
  headerColorTwo=reddishblue,
  headerFontColor=black,
  boxColorOne=lightblue,
  boxColorTwo=lighterblue,
  % Format of textbox
  textborder=roundedleft,
  % Format of text header
  eyecatcher=no,
  headerborder=open,
  headerheight=0.08\textheight,
  headershape=roundedright,
  headershade=plain,
  headerfont=\Large\textsf, %Sans Serif
  boxshade=plain,
  background=shade-tb,
  %background=plain,
  linewidth=2pt
  }
  % Eye Catcher
  {} % No eye catcher for this poster. If an eye catcher is present, the title is centered between eye-catcher and logo.
  % Title
  {\sf %Sans Serif
  %\bf% Serif
  Automatic Topic Discovery in 100 Years of General Conference Talks}
  % Authors
  {\sf %Sans Serif
  % Serif
  Matthew Gardner
  (mjg82@byu.edu)
  and
  Eric Ringger
  (ringger@cs.byu.edu)
  }
  % University logo
  {{\begin{minipage}{16em}
    \hfill
    %\includegraphics[height=5.5em]{logo}
  \end{minipage}}
  }

  \tikzstyle{light shaded}=[top color=baposterBGtwo!30!white,bottom color=baposterBGone!30!white,shading=axis,shading angle=30]

  % Width of left inset image
     \newlength{\leftimgwidth}
     \setlength{\leftimgwidth}{0.78em+8.0em}

%%%%%%%%%%%%%%%%%%%%%%%%%%%%%%%%%%%%%%%%%%%%%%%%%%%%%%%%%%%%%%%%%%%%%%%%%%%%%%
%%% Now define the boxes that make up the poster
%%%---------------------------------------------------------------------------
%%% Each box has a name and can be placed absolutely or relatively.
%%% The only inconvenience is that you can only specify a relative position 
%%% towards an already declared box. So if you have a box attached to the 
%%% bottom, one to the top and a third one which should be in between, you 
%%% have to specify the top and bottom boxes before you specify the middle 
%%% box.
%%%%%%%%%%%%%%%%%%%%%%%%%%%%%%%%%%%%%%%%%%%%%%%%%%%%%%%%%%%%%%%%%%%%%%%%%%%%%%
    %
    % A coloured circle useful as a bullet with an adjustably strong filling
    \newcommand{\colouredcircle}[1]{%
      \tikz{\useasboundingbox (-0.2em,-0.32em) rectangle(0.2em,0.32em); \draw[draw=black,fill=baposterBGone!80!black!#1!white,line width=0.03em] (0,0) circle(0.18em);}}

%%%%%%%%%%%%%%%%%%%%%%%%%%%%%%%%%%%%%%%%%%%%%%%%%%%%%%%%%%%%%%%%%%%%%%%%%%%%%%
  \headerbox{Contribution}{name=contribution,column=0,row=0}{
%%%%%%%%%%%%%%%%%%%%%%%%%%%%%%%%%%%%%%%%%%%%%%%%%%%%%%%%%%%%%%%%%%%%%%%%%%%%%%
   {}
   
   We use state of the art natural language processing techniques to gain new
   insights from 100 years of General Conference Talks.  Using computers to
   sift through a corpus that would be overwhelmingly large to process
   manually, we can discover relationships between talks and trends over time
   that help us to better appreciate the trove of knowledge in these inspired
   discourses.

 }

%%%%%%%%%%%%%%%%%%%%%%%%%%%%%%%%%%%%%%%%%%%%%%%%%%%%%%%%%%%%%%%%%%%%%%%%%%%%%%
  \headerbox{Background and Methodology}
  {name=methodology,column=0,below=contribution}{
%%%%%%%%%%%%%%%%%%%%%%%%%%%%%%%%%%%%%%%%%%%%%%%%%%%%%%%%%%%%%%%%%%%%%%%%%%%%%%
	Document clustering is the task of determining which documents are similar
	to each other in any given corpus.  The fundamental assumption made by
	clustering techniques is that there are several topics used in the corpus,
	and each document is generated by a single (unknown) topic.  Given a
	clustering model and a set of documents, statistical methods can be used to
	determine the best clustering of documents for that corpus.

	When documents frequently cover multiple topics, however, the performance
	of document clustering methods degenerates.  This is often the case with
	religious discourse, as talks frequently combine faith and repentance, or
	faith and tithing, making the talks seem similar in regards to their use of
	words describing faith, but different in the other topics they cover.  To
	solve this problem, Latent Dirichlet Allocation (LDA) was proposed.  LDA
	breaks the assumption of document clustering that each document is
	generated from a single topic, and instead assumes that each word in each
	document comes from a particular topic.  Thus documents can have multiple
	topics, creating a better model for many corpora of interest today.

	LDA can be used both to discover the topics that are spoken of throughout a
	corpus and to determine which documents in the corpus use which topics most
	frequently.  When documents have dates and authors associated with them,
	other interesting attributes of the corpus, such as the relative
	proportions of topics over time, can also be deduced.  We used LDA to
	automatically discover topics found in General Conference talks from
	1900--2005.  We obtained the talks from the internet site
	www.ldslibrary.org.  There is some noise in the data because of parsing
	error, but overall the quality is high.

  }

%%%%%%%%%%%%%%%%%%%%%%%%%%%%%%%%%%%%%%%%%%%%%%%%%%%%%%%%%%%%%%%%%%%%%%%%%%%%%%
  \headerbox{Results}
  {name=results word distributions,column=1,span=2,row=0}{
%%%%%%%%%%%%%%%%%%%%%%%%%%%%%%%%%%%%%%%%%%%%%%%%%%%%%%%%%%%%%%%%%%%%%%%%%%%%%%

	The main result of using LDA is a list of which words are most likely to be
	used in each topic.  In the algorithm, the topics just have numbers; the
	computer merely looks for statistical relationship between words.  The
	labels are thus our own.  We present here a few of the topics we found.

	\small
	\begin{center}
	  \begin{tabular}{ccccc}
		Topic 19: Tithing&Topic 33: Welfare&Topic 52: Women&Topic 37: Families
		&Topic 7: Government\\
		\hline
		tithing&welfare&women&children&government \\
		pay&poor&society&home&united \\
		money&church&relief&family&liberty \\
		debt&fast&sisters&parents&nation \\
		means&help&hinckley&teach&states \\
		out&services&each&child&country \\
		people&program&b&families&freedom \\
		one&plan&gordon&homes&land \\
		paid&care&lives&love&free \\
		poor&needs&service&mother&men \\
		tithes&food&sister&taught&constitution \\
		get&need&forward&fathers&people \\
		more&family&strength&responsibility&america \\
	  \end{tabular}
	\end{center}

	\normalsize
	\mbox{\hspace{0.3\linewidth}\rule{0.4\linewidth}{1pt}\hspace{0.3\linewidth}}\\
	Because each talk has a date associated with it, we can use LDA to analyze
	the distribution of these topics over time, seeing at when certain topics
	were most discussed.  In doing this, we see clear correlations with
	real-world occurrences.

	\begin{center}
	  \includegraphics[height=.34\linewidth]{topic19.eps}
	  \includegraphics[height=.34\linewidth]{topic33.eps}

	  \includegraphics[height=.34\linewidth]{topic52.eps}
	  \includegraphics[height=.34\linewidth]{topic56.eps}
	\end{center}

  }%
%%%%%%%%%%%%%%%%%%%%%%%%%%%%%%%%%%%%%%%%%%%%%%%%%%%%%%%%%%%%%%%%%%%%%%%%%%%%%%
  \headerbox{Future Work}{name=future work,column=3,span=1}{
%%%%%%%%%%%%%%%%%%%%%%%%%%%%%%%%%%%%%%%%%%%%%%%%%%%%%%%%%%%%%%%%%%%%%%%%%%%%%%

	We have used natural language processing techniques on a novel data
	source---LDS General Conference talks.  However, our exploration of the
	results has been quite preliminary.  Our research lays the groundwork for a
	more thorough analysis of religious discourse in the LDS Church.  These
	techniques allow us to quickly answer questions such as, ``Which topics are
	constant over time, and which come and go?'' and ``Which years had the
	highest diversity of topics, and which were more narrow-focused?''

	We also have only considered General Conference talks.  A fruitful avenue
	of future work would be to compare the results of our techniques on many
	different corpora of LDS religious and scholarly texts, such as the Journal
	of Discourses, the Book of Mormon, the Doctrine and Covenants, the
	collected works of Hugh Nibley, or the past 50 years of BYU Studies.  Such
	analysis could broaden our understanding of and appreciation for our
	textual history.

  }%

\end{poster}%
%
\end{document}
