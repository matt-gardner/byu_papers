\documentclass[onecolumn, 12pt]{article}
\usepackage{fullpage}
\usepackage{graphicx}
\usepackage{setspace}
\doublespacing

\title{}
\author{Matthew Gardner}
\date{}

\begin{document}

\textbf{Literature Response}

\begin{tabular}{ll}
  Name:&Matthew Gardner \\
  Work:&\emph{Enoch and the Gorilla} \\
  Author:&Flannery O'Connor \\
  Date Published:& 1971 \\
  Date Read:& January 8, 2010 \\
\end{tabular}

\section*{Historical Context}

Flannery O'Connor was born in 1925 in Georgia.  She was a Catholic living in
the Protestant Bible Belt, and that affected the way that she wrote.  She often
was critical of the secular society around her and felt a return to Catholicism
would be beneficial.  She spent the last 14 years of her life struggling to
fight off lupus, but eventually died of it in 1964, at 39 years of age.
\emph{Enoch and the Gorilla} was published only in 1971, in a collection of
O'Connor's short stories published posthumously under the title \emph{The
Complete Stories}.  However, the story seems to be taken almost directly out of
O'Connor's first major work, a novel called \emph{Wise Blood}, written in 1952.
I was unable to discover why it was published separately as a short
story---perhaps \emph{Wise Blood} started out as several related short stories.
When people were looking through her manuscripts and notes after her death,
they probably found the story on its own and published it with \emph{The
Complete Stories}.

\section*{Critical Analysis}

O'Connor's \emph{Enoch and the Gorilla} seems to be a harsh commentary on the
failures of secular society.  The main character is an 18-year-old boy named
Enoch Emery, who was abandoned by his father and thus forced to live and work
in the city to survive.  The story details a transformation that takes place in
Enoch, and through that transformation reveals the inadequacies of popular
culture.  Through her descriptions of people in the story, the cause of Enoch's
transformation, and its outcome, O'Connor shows secular life to be deficient.

The characters in \emph{Enoch and the Gorilla} reflect O'Connor's negative
views of life in a worldly society.  Every single character in the story is
rude, unkind, or otherwise of miserable character.  From the stingy old
landlady at the beginning of the story that lends Enoch her old, broken
umbrella, to the man in a gorilla suit who tells him to ``go to hell,'' and the
man at the end of the story who wordlessly abandons his girlfriend to be
attacked by a gorilla, all of the people portrayed are seriously flawed, 
including Enoch himself.

Enoch is shown to be lacking confidence in himself and to be quite rude.  He
sees a poster advertising a movie with a gorilla, with a promotion at the
theater featuring the gorilla itself shaking hands with viewers of the movie.
He goes to shake the gorilla's hand, thinking that ``the opportunity to insult
a successful ape came from the hand of Providence.''  While in front of the ape
he gets so nervous that he stammers something largely incoherent, and the ape
turns out to be merely a man in a suit when he gets mad at the boy babbling at
him.  The humiliation Enoch experiences eventually causes a transformation in
him.  He gains self-confidence, and for the first time O'Connor uses the words
``happy'' and ``kind'' in her story, and they are describing Enoch.  The cause
of this transformation, however, was a movie star, an epitome of worldliness.
It seems for a moment that O'Connor is promoting secular living, as it is the
cause of Enoch's newfound happiness.

The outcome of Enoch's transformation tells a different story.  Enoch decides
he wants to be like the gorilla, having people line up to shake his hand.  So
he finds the place the gorilla is, sneaks into the truck, kills the man who was
wearing the suit and takes the suit for himself.  He goes into the woods,
buries his old clothes, puts on the suit, and bounces around growling like a
gorilla.  He literally becomes a beast, both in his murder of the man and in
his insanity in tramping around like a gorilla.  He walks to the highway and
finds a couple sitting on a bench.  He walks up to them to shake their hand,
hoping to find the success he was searching for, but the two run away, leaving
him to stare out into the city, alone again in his misery.  What started out as
a happy transformation turned into another failure, worse than before.

\emph{Enoch and the Gorilla} uses some rather gruesome imagery to provide a
stinging rebuke of secular society.  Even the phrase she uses to describe the
gorilla in the story, ``successful ape,'' hints at how O'Connor feels about
those who are ``successful'' in the secular sense.  The society failed Enoch
and drove him out of it, turning him into a beast in the process and denying
him the happiness he hoped to find.  Likewise O'Connor sees life in a purely
secular world to be meaningless, discouraging, and unable to fulfil our hopes
and our dreams.

\section*{Personal Reflection}

The first time I read \emph{Enoch and the Gorilla}, I was really quite confused
by it.  I had to read some biographical information on Wikipedia about her and
her stories before I began to see any kind of point to the story.  When I read
it again, I could see some meaning, though I really have no way of knowing
whether the meaning I saw was intended by O'Connor or not.  It is for that
reason that I have very mixed feelings about fiction as social commentary.  I
prefer straightforward arguments and commentary, rather than convoluted ones
that are easily misunderstood.  I do appreciate the stories more when I
understand the purpose for which they were written, but most of the time I have
to be told what they mean---it is never readily apparent to me.

\end{document}
