\documentclass[onecolumn, 12pt]{article}
\usepackage{fullpage}
\usepackage{graphicx}
\usepackage{setspace}
\doublespacing

\title{Mendelssohn's Italian Symphony: A Music Response}
\author{Matthew Gardner}
\date{}

\begin{document}

\textbf{Music Response}

\begin{tabular}{ll}
  Name:&Matthew Gardner \\
  Work:&Mendelssohn's Symphony No. 4 (The Italian Symphony) \\
  Artist:&Felix Mendelssohn \\
  Date Composed:&1833 \\
  Date Experienced&10/23/2009 \\
\end{tabular}


\section*{Historical Context}

Felix Mendelssohn was an early Romantic composer who lived from 1809 to 1847.
He went on a tour to Europe from 1829 to 1831, and a number of his compositions
originated on that tour, including his fourth symphony, called ``The Italian
Symphony.''  He wrote the movements of the symphony in response to various
experiences he had in Italy; the second, slow movement is based on a religious
ceremony Mendelssohn viewed in Naples.

The Italian Symphony's original performance was conducted by Mendelssohn in
London, and it was very well received.  However, Mendelssohn was not entirely
happy with it and revised it in 1837, intending even further revisions.  The
symphony was not even published until 1851, after Mendelssohn's death.

\section*{Critical Analysis}

The first movement of ``The Italian Symphony'' conjures images of trotting on a
horse through the Italian countryside on a beautiful day.  The tempo and
rhythms in the piece aren't too fast or energetic, making the ride a leisurely
one.  Before long the music rises to a small climax, as if some big city has
been seen and entered.  After the climax, the music calms a bit and switches to
a number of different instruments all taking turns with new melodies only
slightly related to the theme.  I imagine entering the city and talking with
all sorts of different people.  Occasionally the original theme crops up
throughout this part of the piece, as if we are telling those we are speaking
with about our wonderful journey.

When the solo instruments all quiet down, the theme comes back strong again, as
if to say that we are leaving the city and starting out into the countryside
again.  However, the music quickly turns into a minor key, playing a darker
variation on the original theme.  Something bad has happened in our journey.
Mendelssohn spends some time here developing the struggle, having some major
lines battling with the growing dark mood, but just when it seems that the
journey has gone sour the darkness recedes with a decrescendo to almost
nothing.  A few halting sounds are made in the original major key, which grow
steadily into the theme for our journey.

The movement then starts it conclusion with a flute playing a quick moving line
almost like a dance, interwoven with parts of the original theme.  I imagine a
celebration after escaping whatever caused the dark mood.  There is even a
little bit of the minor melody encroaching on the end, again as if telling a
story about what had happened on the road.  The piece ends with a climax on
the original theme.

The other three movements all contain imagery just as vivid as the first,
though I won't go into as much detail with them.  Knowing that the second
movement was written in response to a religious procession greatly affected the
way I listened to the piece.  The tone was a sort of minor, perhaps eluding to
medieval modes that were used in Catholic services.  It was very solemn, and
there were staccato strings in the background with the sense of walking.  There
were some happy parts in the movement, however, showing that even though their
religion was solemn, it did preach of joy.  There was no exciting climax at the
end of the second movement.

The third movement was also rather mellow, and it seemed almost contemplative
to me.  Perhaps after watching the religious procession Mendelssohn tried to
portray going off under a tree to think about what was just witnessed.  There
were some small variations on the theme of the movement, but all of it was very
calm and smooth.

With the two previous movements being soft and mellow, the opening notes of the
fourth movement are jarring.  The final movement is a combination of Italian
dances, and the first one is exciting and energetic.  It seems like Mendelssohn
varies between two main dances, one that is fast and sensual and one that is
stately and cultured.  Towards the end he gives an effective image of a dance
at court with the violins and the violas taking turns with fast, cheerful runs.
You can really picture the women dancing with the violins, and the men dancing
with the violas.  The short crescendo to the end of the piece gives a nice
ending to a tour through Italy.

\section*{Personal Reflection}

I started out my college experience as a music major, accepted into the program
with a scholarship.  I was intending to study music composition.  But because I
switched my major to computer science, I had not had a deep musical experience
in quite some time.  Reading about this piece and listening for the experiences
that may have inspired Mendelssohn to write what he did was a great experience
for me.  It reminded me of why I love music and wanted to be a composer.  There
are not many pieces of music that I can clearly put imagery to, but
Mendelssohn's Italian Symphony was one of them.

\end{document}
