\documentclass[onecolumn, 12pt]{article}
\usepackage{fullpage}
\usepackage{graphicx}
\usepackage{setspace}
\doublespacing

\title{Mendelssohn's Italian Symphony: A Music Response}
\author{Matthew Gardner}
\date{}

\begin{document}

\textbf{Music Response}

\begin{tabular}{ll}
  Name:&Matthew Gardner \\
  Work:&Mendelssohn's Symphony No. 4 (The Italian Symphony) \\
  Artist:&Felix Mendelssohn \\
  Date:&1833 \\
\end{tabular}


\section*{Historical Context}

Felix Mendelssohn was an early Romantic composer who lived from 1809 to 1847.
He went on a tour to Europe from 1829 to 1831, and a number of his compositions
originated on that tour, including his fourth symphony, called ``The Italian
Symphony.''  He wrote the movements of the symphony in response to various
experiences he had in Italy; the second, slow movement is based on a religious
ceremony Mendelssohn viewed in Naples.

The Italian Symphony's original performance was conducted by Mendelssohn in
London, and it was very well received.  However, Mendelssohn was not entirely
happy with it and revised it in 1837, intending even further revisions.  The
symphony was not even published until 1851, after Mendelssohn's death.

\section*{Critical Analysis}

\section*{Personal Reflection}

\end{document}
