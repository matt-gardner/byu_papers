\documentclass[onecolumn, 12pt]{article}
\usepackage{fullpage}
\usepackage{graphicx}
\usepackage{setspace}
\doublespacing

\title{}
\author{Matthew Gardner}
\date{}

\begin{document}

\textbf{Film Response}

\begin{tabular}{ll}
  Name:&Matthew Gardner \\
  Film:&\emph{Searching for Bobby Fischer}\\
  Production Year:&1993 \\
  Language:&English \\
  Director:&Steven Zaillian \\
  Country of Origin:&USA \\
  Location:&HBLL \\
  Date Attended:&12/28/2009 \\
\end{tabular}

\section*{Historical Context}

\emph{Searching for Bobby Fischer} is based on a true story about chess prodigy
Josh Waitzkin.  Josh becomes interested in chess at age 6 when he starts
playing speed chess in a park.  He gets a professional teacher and enters
tournaments, winning the national junior championship at the end of the movie.
The title references the only American to ever have claimed the World Chess
Championship, Bobby Fischer.  Josh's teacher and others in the movie make
frequent reference to Fischer, and Fischer's attitude towards chess plays a
central role in the plot.  The movie is based on a book written by Josh's
father, who told the story of Josh's chess playing.

\section*{Critical Analysis}

In \emph{Searching for Bobby Fischer}, Zaillian sets up a series of conflicts
between life and chess, having fun and winning.  Throughout the movie, two of
the main characters, Josh's dad Fred and his teacher Bruce, go through a 
transformation from caring only about being the best to caring about people and
enjoying life.  We see this conflict and transformation in a number of scenes,
starting with Fred looking for a teacher for his son and ending with the 
national tournament.

Towards the beginning of the movie, Fred realizes that his son has a gift for
playing chess and seeks a teacher for him.  He finds Bruce, who initially tries
to dissuade Fred by showing him a tournament with a bunch of chess fanatics,
one of whom spends his whole life thinking about nothing but chess, and earns
\$2,000 a year.  Bruce's point is that to win at chess, you have to give your
life to it.  Fred is persuaded, and he takes that attitude throughout most of
the rest of the movie.  

Fred enters Josh in tournaments, and Josh quickly becomes the top-rated junior
player in the state of New York.  The conflict raises its head again at an open
house for Josh's school, where his teacher talks to Fred about her concern that
Josh is falling behind in his school work and his friendships over ``this chess
thing.''  Fred is irate and storms out with Josh, offended enough that he moves
Josh to a private school where they have chess classes.  Chess, and his son
winning chess tournaments, has become Fred's obsession far more than Josh's.

The height of the conflict between winning and having fun comes during the
state finals.  Josh doesn't really want to go, because he's afraid he might
lose, but his dad tells him, ``You won't lose.''  He goes to the tournament and
loses, quickly, upon which his dad takes him to a rainy park and berates him
for throwing the match.  Josh's childhood is lost because all his dad cares
about is that he wins.

Shortly after the match, Bruce tries to get Josh to play more like Bobby
Fischer, holding the opponent in contempt.  Josh is an innocent kid who just
wants to have fun, and he doesn't like the idea of turning chess into a war.
Bruce pushes him, and Josh's mom kicks him out.  When Fred sides with Bruce,
she threatens divorce to get Josh out of a situation that would ruin him.  This
seems to be the turning point for Fred.

When Fred realizes that he has gone too far in his desire to see Josh win, he
starts taking things easy.  Josh still plays chess, but only with the people in
the park.  They go to the national championship, and though Josh wants to win,
he and his dad are not stressed out about it anymore.  Fred even takes Josh
fishing for two weeks before the tournament and tells Josh he's not allowed to
even think about chess.  Bruce even comes around in the end, coming to the
championship even though he initially told Fred that he wouldn't.  He
apologizes to Josh and wishes him the best.  

In the beginning of the movie Bruce had told Josh that chess was an art to
Bobby Fischer.  It is to Josh too, and it is his love for the game, not his
love for winning, that ends up winning him the national tournament.  Much to
Bruce's chagrin, Josh uses a style that he learned playing speed chess in the
park in his final match---his art.  But he also utilized the analytical skills
taught him by Bruce.  When he sees that he has won the game, Josh offers his
opponent, a chess fanatic, a draw.  The other boy refuses, not realizing that
he has lost, and Josh wins the championship.

Though the movie ostensibly is about chess, the entire plot seems to center
around this struggle between enjoying life and wanting to win.  The director
clearly shows which he thinks is better by having the happy, loving, and
playful Josh win the final tournament over the boy who devotes his entire life
to chess.

\section*{Personal Reflection}

I really enjoyed watching \emph{Searching for Bobby Fischer}, for a number of
reasons.  First, I quite enjoy the game of chess, though I'm no prodigy, so the
subject matter was interesting to me.  Second, it made me think a lot about
parenting and the best way to raise children, something that is particularly
applicable to me at the moment because I will be a father in less than two
months.  And lastly, I have thought a lot about wanting to be the best, to win.
If you want to be the best in the world at anything, you pretty much have to
devote your life to it.  No matter what your innate capacity, someone with
similar capacity is going to be devoting their life to that thing, so in order
to compete with them you have to also devote similar time.  Is being the best
in the world, or even the best in your school, really worth that kind of time?
In almost every case, I don't think it is.  There are far too many things to do
in life to worry about being the best at any particular talent.  So, I always
try to do \emph{my} best, but I don't worry too much about being \emph{the}
best.

\end{document}
