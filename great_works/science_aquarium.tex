\documentclass[onecolumn, 12pt]{article}
\usepackage{fullpage}
\usepackage{graphicx}
\usepackage{setspace}
\doublespacing

\title{}
\author{Matthew Gardner}
\date{}

\begin{document}

\textbf{Life Science Response}

\begin{tabular}{ll}
  Name:&Matthew Gardner \\
  Site:&Aquarium of the Pacific \\
  Location:&Long Beach, California \\
  Date Established:&1998 \\
  Date Visited:&12/17/2008 \\
\end{tabular}

\section*{Historical Context}

The Aquarium of the Pacific opened to the public in 1998.  It has three major
areas that each highlight a different habitat of the Pacific Ocean---Southern
California and Baja, the Northern Pacific, and the Tropical Pacific.  It also
has smaller exhibits that change periodically.  I grew up in Southern
California and made several trips to the aquarium when I was younger.  More
recently, I visited it with my wife over Christmas break in 2008, and each time
the experience has been fascinating for me.

\section*{The Experience}

The first part of the experience that was enlightening to me was seeing the
incredible diversity of life that exits in the ocean, most of it largely unseen
to us on the land.  There's an odd beauty to a large bloom of jellyfish, or a
motley collection of exotic marine life.  One of my favorites is the
humuhumunukunukuapua'a, or trigger fish, which is the state fish of Hawai'i.
Going to zoos or aquariums always gives me an incredible appreciation for the
beauty and complexity of God's creations.

Along with the beauty of the creatures on display at the aquarium, the rarity
of many of the animals struck me in a couple of ways.  First, there were some
animals that live at the bottom of the ocean, where relatively little
exploration has taken place.  There were some giant crabs that seemed massive
to me, as well as some really strange looking seahorses.  Seeing those animals
made me wonder about what other forms of life exist in the deep ocean that we
know only very little about, like the giant squid, deep sea angler, or
long-nosed chimaera.  Not only is there beauty in God's creations, there's a
breadth that we have only begun to explore.

The rarity of the animals also made me think about how we are impacting our
environment, and whether or not that will adversely affect the habitats of
these creatures.  Some, it seems, are common and widespread enough that any
relatively minor change in the environment wouldn't do them too much harm.
Others seem delicate and exotic, and destroying the one reef where they are
found might see them completely disappear.  If it turns out that the earth does
warm significantly, many of these creatures may some day exist only in pictures
and memories.  That would be a sad day indeed.  However, history is replete
with examples of animals that failed to adapt to their environment and became
extinct---the adaptability of life is one of the wonders of God's creations,
even if it means some species must die out to make room for new ones.

My trip to the Aquarium of the Pacific filled me with both a child-like wonder
at the beauty all around us and a feeling of responsibility to treat our
environment a little nicer.  It was a great experience.

\end{document}
