\documentclass[onecolumn, 12pt]{article}
\usepackage{fullpage}
\usepackage{graphicx}
\usepackage{setspace}
\doublespacing

\title{}
\author{Matthew Gardner}
\date{}

\begin{document}

\textbf{Theater Response}

\begin{tabular}{ll}
  Name:&Matthew Gardner \\
  Play:&The Scarlet Pimpernel \\
  Playwright:&Frank Wildhorn \\
  Date Written:&1997 \\
  Date Seen:&September 1, 2009 \\
  Location Seen:&Hale Center Theater, Salt Lake \\
\end{tabular}

\section*{Historical Context}

\emph{The Scarlet Pimpernel} was originally a play Baroness Emmuska Orczy,
produced in 1903.  Orczy wrote a novel out of it two years later.  It has since
seen many film and television adaptations and was made into a Broadway musical
in 1997 with some minor changes to the storyline.  It is set in England and
France during the French Revolution.  Sir Percy Blakeney, an English nobleman,
marries a French actress, Marguerite St. Just.  Percy is disgusted by the
brutality of the French Revolution, and decides to begin a personal crusade
against it, calling himself the Scarlet Pimpernel.  Thinking his wife is a
French spy, he hides his identity from her and everyone else except his secret
band by acting like a fop.  He and his friends end up saving the lives of many
Frenchman and manage to frame Citizen Chauvelin, the French agent charged with
finding him, as the Scarlet Pimpernel.

\section*{Critical Analysis}

\emph{The Scarlet Pimpernel} is a very moving story.  The music and backdrop
for one of the opening songs, ``Madame Guillotine'' does an incredible job at
capturing the terror and brutality of the French Revolution.  Shortly
afterward, Percy sings a stirring challenge to those who become his band to not
sit idly by and let this great evil go unhindered.  There are some very powerful
moments in the play.  But woven throughout \emph{The Scarlet Pimpernel} is the
recurring use of disguises, both as a necessary protection to those fighting
evil and as the unfortunate consequence of a failure to communicate.

The play opens with Marguerite singing in her final show before leaving to be
married.  She sings of fairy tales and storybooks, setting up a theme that is
touched on throughout the play on two levels---she is acting, and she is acting
about false identities.  But her acting turns out to be her downfall.  Before
she met Percy, she had been a spy for Chauvelin during the beginnings of the
revolution.  Chauvelin uses that to blackmail her into betraying a friend of
both her and Percy, and when Percy learns of her betrayal he thinks she is
still a spy.  Thus Percy withdraws his love and hides his escapades from her,
leaving her lonely and confused about why her husband changed so dramatically.

Percy and his band also use disguises, though for different reasons.  In order
to protect themselves from the French they have to hide their true identities.
They do so in a very odd manner, pretending to be nincompoops and fops.  This
guise reaches its apex when they parade their fancy frillery in front of the
prince, singing, ``Each species needs a sex that's fated to be highly
decorated.  That is why the Lord created \emph{men}.''  Because they are so
focused on fashion, no one suspects that they really are the band of the
Scarlet Pimpernel.  They effectively use disguises to thwart many executions of
innocent people in France.

Percy's disguise does have its problems, however.  As mentioned earlier, Percy
alienates his wife with his disguises.  In the final song of Act I, ``The
Riddle,'' Percy, Marguerite and Chauvelin sing about the perils of having too
many disguises, not knowing who to trust.  ``And we all have so many faces,''
they sing, ``the real self often erases.  Enticing lies flicker through our
eyes.''  Percy wants to trust his wife, Marguerite is torn between trusting her
husband and going back to Chauvelin, Chauvelin wants to trust Marguerite, but
none of them can see through the disguises to the real motives underneath.

Hidden identities play a central role in \emph{The Scarlet Pimpernel}.  At
least five of the songs in the musical have disguise or concealment as their
focus.  Through this imagery Wildhorn attempts to convey the loss and confusion
that disguises can bring, in both good and bad ways.  Percy and Marguerite
eventually learn the truth about each other and are reconciled, but they lost a
lot of time together because of the barriers they built between each other.
And Chauvelin in the end gets the just reward of his treachery---Percy and his
gang capture him and leave him looking like the Scarlet Pimpernel, to be found
and surely executed by the leaders of the French Revolution.


\section*{Personal Reflection}

I first saw \emph{The Scarlet Pimpernel} when I was about 14, and I have loved
it ever since.  What was particularly striking to me when I saw it at the Hale
Center Theater was the feeling of sheer terror that the cast created during the
beginning, while singing ``Madame Guillotine.''  I was on the second row, and
they brought the ``Reign of Terror'' in the French Revolution to life.  It made
Percy's song about fighting injustice and not letting evil go unchallenged all
the more powerful to me, as it was a very real terror that they were fighting.
Though perhaps on a smaller scale, are we not to do the same in our lives
today?  I also was struck by all of the grief caused to both Percy and
Marguerite because they were quick to assume the worst in each other.  If they
had clearly communicated and resolved their differences, they would have had a
much happier marriage early on.  It is seeing tragedies like the unnecessary
sorrow between Percy and Marguerite that make me try especially hard to be open
and clear in my relationship with my wife.

\end{document}
