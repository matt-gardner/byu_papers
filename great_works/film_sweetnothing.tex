\documentclass[onecolumn, 12pt]{article}
\usepackage{fullpage}
\usepackage{graphicx}
\usepackage{setspace}
\doublespacing

\title{}
\author{Matthew Gardner}
\date{}

\begin{document}

\textbf{Film Response}

\begin{tabular}{ll}
  Name:&Matthew Gardner \\
  Film:&\emph{Sweet Nothing in My Ear} \\
  Production Year:&2008 \\
  Language:&English and American Sign Language \\
  Director:&Joseph Sargent \\
  Country of Origin:& USA \\
  Location:&International Cinema \\
  Date Attended:&11/18/2009 \\
\end{tabular}

\section*{Historical Context}

\emph{Sweet Nothing in My Ear} debuted in 2008 as a Hallmark Hall of Fame
movie.  The movie is based on a play by the same name written by Stephen Sachs
in 1998.  It is a story about a married couple, Dan and Laura Miller.  Laura is
deaf, as are her parents, and they are all very committed to deaf culture.
They have a son, Adam, who was born hearing, but gradually lost his hearing
after a few years.  When the father decides he wants to try giving his son a
cochlear implant, it causes marital problems, because the wife doesn't see
herself as handicapped and doesn't see anything wrong with her son.  They
separate, and the movie is shown as a series of flashbacks during a court
battle for custody of the boy.  The film focuses on the struggle of minority
groups and their attempts to define their identity in society.  The director,
Joseph Sargent, directed a movie with a similar theme 35 years ago titled
\emph{Love is Never Silent}, using some of the same actors.

\section*{Critical Analysis}

While the main point the director attempts to discuss is minority groups trying
to find their place in society, \emph{Sweet Nothing in My Ear} also highlights
the pain caused by pride, particularly in family struggles.  Throughout the 
movie we see children caught in the middle of fights between their parents,
damaging everyone involved.  We see this in Laura's parents, in harm caused to
Adam, and between Laura and Dan.

Laura's father, Max, was born deaf.  He was ridiculed as a child and appears to
have developed a complex because of it, thinking that being deaf is better than
being hearing.  He wrote a book on deaf pride and the deaf culture.  In one of
the most climactic scenes in the movie, he tells Laura that she wasn't actually
born deaf, something he had lied about to her all her life.  His pride in his
deafness had caused him to hide the truth from her, and she is deeply hurt when
he finally tells her she was born hearing and only gradually developed
deafness.

Much of the same pride seen in Max is found in his daughter Laura, and the
fights between Laura and her husband Dan leave their son Adam as the victim.
In scene after scene in the movie, we see Adam wondering why his parents are
fighting, wanting both of them to be together again, and wondering why his dad
isn't home with them.  In the climax of the movie, during Thanksgiving dinner
with his parents and Laura's parents, Adam tries to please his dad by speaking
to him.  He says ``Daddy'' in a halting voice that hasn't been used in years.
Dan is jubilant, but Laura and her parents are shocked, feeling betrayed.  The
phone rings, and Dan goes to get it, while Laura and her parents get up to talk
in another room.  In what to me was the most poignant scene in the movie, Adam
is left at the dinner table by himself, wondering what went wrong.  In the
power struggle between his parents, Adam is the one who is hurt the most.

While it was Adam who suffers the most from his parents' fighting, Laura and
Dan aren't left unscathed.  Their pride and unwillingness to come to a
compromise ruins their marriage and leaves them both miserable.  After deciding
without consulting his wife that he would look into getting Adam a cochlear
implant, Dan begins scheduling meetings for the two of them behind Laura's
back.  Neither one is willing to let go of what they want, and their arguments
get progressively more bitter.  It all culminates in a verbal fight they have
one night at home where Dan says to Laura, ``God gave me a hearing son!'' and
Laura asks why Dan thinks she isn't good enough because she can't hear.  In the
end of the movie, they both realize the harm they have caused each other and
their son with their pride, and Dan repentantly comes back to Laura, seeking
her forgiveness.

\emph{Sweet Nothing in My Ear} does a masterful job describing the conflict
between minority and majority groups.  Those conflicts are often caused by
pride, and Joseph Sargent poignantly portayed the damage done by parents in
their power struggles.  Family life can only be happy when parents are willing
to set aside their pride and do what's best for their family.

\section*{Personal Reflection}

I have had some experience with American Sign Language, though not very much
with deaf culture.  I had some friends in high school who took ASL classes, and
I taught a deaf man the gospel on my mission through an interpeter.  I have
also watched my sister teach her young children sign language so they can
communicate before they are able to speak.  It was fascinating to me to see
more of a world I have only had very limited contact with.  It made me
appreciate more the struggles that parents go through in trying to raise their
children, especially when disabilities are involved.  And as I am fairly
recently married and my wife is expected our first child, it made me think
about what it will be like to try to raise a family, and made me grateful that
my wife and I are able to peacefully resolve our conflicts.

\end{document}
