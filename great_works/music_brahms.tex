\documentclass[onecolumn, 12pt]{article}
\usepackage{fullpage}
\usepackage{graphicx}
\usepackage{setspace}
\doublespacing

\title{}
\author{Matthew Gardner}
\date{}

\begin{document}

\textbf{Music Response}

\begin{tabular}{ll}
  Name:&Matthew Gardner \\
  Work:&Piano Concerto No. 1 in D Minor, Op. 15 \\
  Artist:&Johannes Brahms \\
  Date:&1858 \\
\end{tabular}

\section*{Historical Context}

Johannes Brahms lived from 1833 to 1897 and was one of the central figures in
the Romantic period of music.  He was heavily influenced by earlier periods,
especially fellow German composers Johann Sebastian Bach and Ludwig van
Beethoven.  Brahms was a virtuoso pianist and many of his works were piano
pieces.  Quite often he gave the initial performance of his piano works, 
including his first piano concerto.

Brahms started work on his first piano concerto in 1854, when he was 20 years
old.  It was originally intended to be a sonata for two pianos, then turned
into a symphony.  When he found the symphony to be unacceptable, he settled on
making it a concerto.  He finished the work in 1859 and debuted the work
himself.  The initially performance was incredibly poorly received---Brahms was
hissed at by the audience.  However, over time, the work has come to be
considered a masterpiece.

\section*{Critical Analysis}

\section*{Personal Reflection}

\end{document}
