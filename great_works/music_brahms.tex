\documentclass[onecolumn, 12pt]{article}
\usepackage{fullpage}
\usepackage{graphicx}
\usepackage{setspace}
\doublespacing

\title{}
\author{Matthew Gardner}
\date{}

\begin{document}

\textbf{Music Response}

\begin{tabular}{ll}
  Name:&Matthew Gardner \\
  Work:&Piano Concerto No. 1 in D Minor, Op. 15 \\
  Artist:&Johannes Brahms \\
  Date Composed:&1858 \\
  Date Experienced:&11/5/2009 \\
\end{tabular}

\section*{Historical Context}

Johannes Brahms lived from 1833 to 1897 and was one of the central figures in
the Romantic period of music.  He was heavily influenced by earlier periods,
especially fellow German composers Johann Sebastian Bach and Ludwig van
Beethoven.  Brahms was a virtuoso pianist and many of his works were piano
pieces.  Quite often he gave the initial performance of his piano works, 
including his first piano concerto.

Brahms started work on his first piano concerto in 1854, when he was 20 years
old.  It was originally intended to be a sonata for two pianos, then turned
into a symphony.  When he found the symphony to be unacceptable, he settled on
making it a concerto.  He finished the work in 1859 and debuted the work
himself.  As it initially started out as a symphony, it is very long---at about
50 minutes, it is one of the longest concertos ever written.  The initial
performance was incredibly poorly received---Brahms was hissed at by the
audience.  However, over time, the work has come to be considered a
masterpiece.

\section*{Critical Analysis}

The first movement begins with a long orchestral introduction, which is not
surprising, given Brahms' original intent of writing a symphonic work.  After
the introduction, the piano continues to develop the themes introduced in the
introduction.  As the piece progresses, there is a lot of back and forth
between the piano and the orchestra.  The orchestra takes a central role in the
piece, developing themes and often playing without the piano, instead of merely
acting as accompaniment for the main performer, as is often the case in a
concerto.

Some pieces of music capture a piece of artwork, or a scene, and conjure images
of events to the listener; Mendelssohn is famous for his vivid compositions.
Other pieces of music are more cathartic, simply taking you through emotions
instead of bringing actual images to your mind.  To me, this piece was more of
the second type.  There were moments of intense conflict followed by sweet
resolution, and swells that rise and ebb and take you along with them.  The
movement and dynamic contrast throughout the piece elicit a powerful emotional
response.

There are several parts throughout the first movement, and the entire piece,
that feature either the horn or the timpani.  Those instruments are notoriously
difficult to write for, and Brahms does a masterful job as using those
instruments. 

The first movement is almost 25 minutes long, leaving a lot of time for
development of the original theme.  I was amazed at how well Brahms kept me
engaged throughout the piece.  There was enough variation in the theme that the
music always seemed new and interesting to me, but it was similar enough to
what came before that I was always reminded of the original theme.

The second movement was slower and calmer, as is typical in multi-movement
pieces like symphonies and concertos.  It gave a stark contrast with the first,
energetic movement.  Throughout the piece, the music was peaceful, though the
occasional discord leaves the listener with something to think about and
retains the listener's interest.  The movement does not have very much dynamic
contrast, most of the piece being soft, though Brahms compensates for that with
beautiful melodic lines, particularly in the woodwinds.  The ending of the
movement decrescendos down to almost nothing, with a lone timpani playing the
final notes---another example of Brahms' fine use of the timpani.

The final movement is a rondo, which are characterized by an oft-repeated
central theme.  This rondo is no different, and the energetic theme is
introduced right at the beginning in the piano.  The orchestra immediately
echoes the theme.  Throughout the rest of the piece, it is almost always the
orchestra that repeats the theme, while the piano plays complex variations on
it.  The piece ends in the typical fashion with a large crescendo to a final 
climax.

\section*{Personal Reflection}

Brahms' first piano concerto was a great experience for me a listen to,
especially as I listened to it again to find things I missed the first time.
The more I listen to music the more I notice in it, and it gives me a greater
appreciation for the skill of both the composer and the performers.  I found
the first movement of the concerto incredibly intriguing, especially the amount
of variation that Brahms' could come up with while still basing the piece on a
single theme (a point I mentioned earlier).  Because the piece was so long,
however, I found myself losing interest during the middle of the second
movement.  Perhaps that is why Brahms' original audience hissed at him during
the performance.

\end{document}
