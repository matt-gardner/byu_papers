\documentclass[onecolumn, 12pt]{article}
\usepackage{fullpage}
\usepackage{graphicx}
\usepackage{setspace}
\doublespacing

\title{Children of Eden: A Theater Response}
\author{Matthew Gardner}
\date{}

\begin{document}

\textbf{Theater Response}

\begin{tabular}{ll}
  Name:&Matthew Gardner \\
  Play:&Children of Eden \\
  Playwright:&Steven Schwartz \\
  Date:&1991 \\
\end{tabular}

\section*{Historical Context}

``Children of Eden'' began as a production for a religious high school theater
camp.  It was originally titled ``Family Tree'' and was a short play.  In 1990,
Steven Schwartz, composer and lyricist of the recent hit musical ``Wicked,''
took the original script and music of ``Family Tree'' and made it into a full
length musical, renaming it to ``Children of Eden.''  The play depicts the
early history of the world, starting with God creating the earth, along Adam
and Eve, going through Adam and Eve's expulsion from the Garden of Eden, and
ending in act one with Cain killing Abel.  Act two tells the story of Noah.
The play ends with Noah and his family getting off of the ark and singing
praise to God.

``Children of Eden'' opened in London in 1991.  It received poor reviews and
only stayed open for three months.  Since 1991, the show has never made it to
Broadway, though it has been performed by amateurs and professionals all over
the country.  Its appeal comes from its religious nature and its treatment of
themes common to all people, such as love, anger, and family conflicts, as well
as its ability to accommodate any reasonable cast size.

\section*{Critical Analysis}

Perhaps the central theme of the entire play is the interaction of parents with
their children, particularly their disobedient children.  In the three separate
stories told in the show, very similar dialogue is used between the father and
his children.  Through these parallels, Schwartz explores the proper role of
parents in the lives of their children, and how a father grows through his
experience of being a parent.

The first parent-child story told is that of God, called Father, and Adam and
Eve.  God is the archetypical father, portrayed more as a human father than a
God, lacking many of the qualities we normally ascribe to him.  He seems to be
learning how he should be treating his children, and when they disobey him, Eve
eating the fruit and Adam choosing to follow her, he becomes spiteful and
petty.  He refuses to ever speak to Adam and Eve again, visiting their children
while ignoring them.  At this point in the story, Father still has a lot to
learn about being a father.

The play then moves to Adam and Eve raising their two children, Cain and Abel.
Cain sees a waterfall on a hill and asks his parents about it, using almost
exactly the same words as Eve used when she asked Father about the Tree of
Knowledge.  The parallel dialogue continues throughout the story; Schwartz
clearly tries to set up a comparison between Eve and Cain, and Father and Adam.
In this case again, Adam fails as a parent, Cain kills Abel, and sorrow comes
to the family.

Things look just as bleak when the story of Noah starts.  Noah's son, Japheth,
has yet to choose a wife, and God forbids Noah from letting Japheth marry a
woman of the race of Cain.  Japheth ends up choosing the serving girl, who is a
descendant of Cain, and sneaking her on the ark, upsetting God and the whole
family.  Again the dialogue between Japheth, the servant girl, and Noah is very
similar to the parent-child dialogue in the earlier stories.  God, at this
point the epitomy of a petty grudge-holder, literally turns his back on the
world, including Noah.  But Noah realizes that his family is all that he has,
and forgives Japheth and allows him to marry the servant girl.  Finally, a
parent in the story learned to forgive, and apparently Noah taught God the same
lesson, for they both sing about how the most important part of love is ``the
letting go.''

The parallels that Schwartz creates are clearly meant to be a commentary on the
proper role of a parent in the life of his or her child.  Through the three
stories that Schwartz uses, the audience can see God, the quintessential father
figure, learning how to treat his children.  Likewise, we are meant to learn
through Schwartz's examples how we can be better parents to our children.

\section*{Personal Reflection}

My wife and I are expecting our first child in three months.  The opening scene
of the play, portraying God very much as any loving earthly father, made me
think a lot about how having children teaches us about God.  Perhaps that is
one of the main reasons that God commands us to live in families and have
children---it teaches us about how He feels towards us.  I didn't enjoy the
rest of the play nearly as much, as I don't believe God to be so petty (and my
wife didn't appreciate the parallel between Eve and Cain).  But when I realized
that Schwartz was commenting more on how we should treat our children than the
specifics of a perfect God's interaction with man, I gained more appreciation
for the story that was told.

\end{document}
