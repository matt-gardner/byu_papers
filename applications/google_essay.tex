\documentclass[onecolumn, 11pt]{article}
\usepackage{fullpage}
\usepackage{graphicx}

\title{The Impact of a Simple Example}
\author{Matthew Gardner}
\date{\today}

\begin{document}
\maketitle
I was in a car barreling down a Mexican highway with my family on a trip to see
the Mayan ruins in Yucatan.  As can only happen in the tropics, a storm came
out of nowhere, and the rain was so hard my aunt pulled to the side of the
road.  As we were waiting for the storm to subside, one of the highway signs
fell over right in front of us, with a large part of it sticking out into the
road.  We saw cars go zooming by, some of them barely swerving in time to miss
the fallen sign.  Someone was going to hit the sign and wreck their car and
die; we were sure of it.  But no one was going to do anything about the sign,
and no government workers were on their way to fix it.  So my brother and I
jumped out of our car, ran through the rain to the sign, and started pulling it
out of the road.  We didn't get it very far.  But a few other people, when they
saw us pulling the sign, stopped to help us get the sign out of the way.  With
their help we were able to get the sign onto the shoulder, where passing cars
weren't in danger of hitting it.

I've reflected a lot on that experience.  No one stopped to move the sign while
we were sitting in our car watching.  They probably thought that someone should
move it, but didn't think of doing it themselves.  It wasn't until they saw
someone doing something about it that they started to stop and help.

An odd coincidence to me was that at the time I was reading in the book 1776,
by David McCullough.  As we drove away from the highway sign, I read about
Thomas Paine writing his pamphlet, "The American Crisis."  He said, "These are
the times that try men's souls: The summer soldier and the sunshine patriot
will, in this crisis, shrink from the service of his country; but he that
stands by it now, deserves the love and thanks of man and woman."  George
Washington's army was deserting him, thinking that the cause of America had
failed.  Thomas Paine's words, and his courage in the face of great opposition,
inspired the men to return to Washington and continue fighting.

I think in all of us there is a pull to be something greater than we are, to do
great things.  To be noble and good.  Most of the time we're too afraid or too
busy or too lazy to do anything about it.  But when we see someone else doing
great things, it gives us courage to step up and be the people we know we
should be.

What is it that gives a man or a woman the courage to be a leader?  I don't
know, except that he or she was probably inspired by someone else first. Ever
since my epiphany on a Mexican highway, I've tried to be the kind of person
that inspires others to be great.  

For example, I had a roommate not too long ago that was quick to get upset and
argue with me and my other roommates.  I was often the calming influence in
those arguments, helping everyone to act civilly with each other.  Eventually,
through my example of not getting upset, the atmosphere of the apartment
changed, and there was a lot less arguing.  It's a small thing, I know, but
character is defined much more by small, daily actions than one heroic feat.
\end{document}
