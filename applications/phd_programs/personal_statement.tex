\documentclass[onecolumn, 12pt]{article}
\usepackage{fullpage}
\usepackage{graphicx}

\title{}
\author{}
\date{}

\begin{document}

\pagestyle{empty}

\begin{center}
  \textbf{Personal Statement}\\
  Matthew Gardner
\end{center}

It was spending time as a Spanish-English interpreter that first piqued my
interest in natural language processing.  What if a computer could interpret
accurately and in real time between two languages?  What problems would have to
be solved in order to build such a system?  These questions fascinated me, and
I set out to spend my undergraduate career learning as much as I could about
their answers.

In my second year of college I started as a research assistant in the Applied
Machine Learning lab at Brigham Young University.  The project that was given
to me to work on was particle swarm optimization (PSO).  While my work in PSO
did not relate directly to my interest in machine interpretation, it taught me
what it is like to do real research, as well as the joys and sorrows of the
paper submission process at conferences and journals.

All told I spent two and a half years working on various aspects of PSO.  At
the beginning I was just running experiments with code other people wrote, a
simple bookkeeper.  I progressed to helping analyze and interpret the results,
then to writing a little code of my own.  After about a year in the lab, I got
my first publication; I was a co-author on a paper about how parallelization
affects how PSO~\cite{mcnabb-2009-large-particle-swarms}.  

Motivated by the success of our parallel PSO paper, we spent more effort on how
to better parallelize the algorithm.  My advisor wondered if PSO could be
speculatively decomposed, allowing extra processors to do two or more
iterations of PSO at the same time.  I thought that was an interesting idea and
I made it my honors thesis, moving the idea to an algorithm and then to an
implementation.  After testing my new method and discovering that it
outperformed standard techniques, I turned my work into a paper and
successfully defended my honors thesis.  Confident that my work was amazing, I
submitted my thesis to one of the top conferences for PSO research, fully
expecting a glowing reception.

I had my first major setback when that paper was rejected outright.  At this
point I had spent two years working on PSO, and I felt like giving up and
moving on.  PSO wasn't my main interest anyway, and I wanted to do more
interesting things.  My advisor, however, insisted that I perservere, telling
me that the experience would be good practice for a PhD.  So I revised my paper
and submitted it to another top conference, where it was
accepted~\cite{gardner-2010-speculative-evaluation-in-pso}.

In doing the work for my honors thesis I had many ideas for how to improve the
method I pioneered.  Once I recovered from the shock of my first rejection, I
spent some time exploring new methods that gave significantly greater
performance than my original implementation.  I expanded my honors thesis with
these new ideas and submitted my work to the journal \emph{Swarm Intelligence},
where it is currently under
review~\cite{gardner-2010-speculative-approach-to-parallelization-pso}.

My work in particle swarm optimization was not directly applicable to this
program to which I am currently applying, though a general knowledge of
optimization is certainly useful in many fields of computer science.  However,
I feel that the experience I had working on that problem has done a great deal
to prepare me to succeed in your program.  I have experience conducting real
research, I know how to bring ideas to fruition, and I have successfully taken
my own innovations and had them recognized by the wider research community.
I am confident that, while there will inevitably be setbacks in any research
endeavor, I can perservere and succeed.

\bibliographystyle{plain}
\bibliography{bib}

\end{document}
