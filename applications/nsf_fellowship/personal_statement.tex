\documentclass[onecolumn, 12pt]{article}
\usepackage{fullpage}
\usepackage{graphicx}
\usepackage{natbib}

\setlength{\textheight}{9in}

\begin{document}

\pagestyle{empty}

\begin{center}
  \textbf{Personal Statement}\\
  Matthew Gardner
\end{center}

My father, grandfather, and great-grandfather all earned PhDs in physics.  I
grew up wanting to be a physicist, to win the Nobel Prize by discovering a
Unified Field Theory.  So of course, when I started college, I chose physics as
my major.  During my freshman year I was actively involved in the Society of
Physics Students, encouraging other students to study physics, helping to tutor
lower-level students, and even giving physics and chemistry demonstrations at a
local middle school, hoping to excite the children to learn about science.

I also took a couple of computer science classes my freshman year, as the field
was interesting to me, though physics and mathematics still dominated my
thoughts.  It wasn't until I took a break from school for two years and had
some time away from classes that I really started to be fascinated by computer
science.

After my freshman year of college, I spent two years as a volunteer missionary
in Alabama, paying my way to serve the hispanic community in that state.  I
learned Spanish in order to communicate with them, and I came to understand
much better the circumstances that immigrants to our country live in.  I did a
significant amount of Spanish-English interpretation, and it was in that
capacity that I first thought about the potential for computer science to
improve the lives of ordinary people.  What if a computer could interpret live
speech from one language to another, accurately and effectively?  Such a system
would ease the burdens on so many immigrants who have great difficulty
communicating with other people for their daily needs, and not just those in
our country.

I came back from my experience in Alabama with a new-found desire to study
computer science.  I changed my major to computer science, with minors in
physics and mathematics, as those subjects were still interesting to me and
they are relevant background for a computer scientist.  I found a computer
science professor who was doing research that I found intriguing; I wanted to
learn about machine translation, which I knew was part of machine learning, so
I started working in the Applied Machine Learning lab with Dr. Kevin Seppi.

My time working in the Applied Machine Learning lab has been the defining
experience of my undergraduate education.  I had very close interaction with
several professors and many graduate students who showed me how interesting
research can be.  From those that I had association with I gained a love of
learning new things and exploring new ideas.  I also learned a great deal more
than was taught in my classes in order to understand the research that was
going on in the lab.  Several times in my classes the professor teaching the
class had me give part of a lecture that dealt with things I was actively
researching.  

My undergraduate education would not have been nearly so fulfilling and
enjoyable without my experience working in a research lab.  My understanding of
the ideal college experience has been fundamentally changed through my
research.  As a professor, which I hope to be some day, one of my goals will be
to help all of the undergraduates in my program have the same kind of
experience that I had.  Working closely with professors and graduate students
and conducting independent research were wonderful opportunities that every
undergraduate student should be able to enjoy.

I also had a great internship experience as an undergraduate that helped to
shape my vision of what I want from my career.  I worked as an intern on the TV
Ads team at Google during the summer of 2009.  My intern host didn't have a
clear idea for the project she wanted me to do, so I proposed a meaningful,
difficult project that, if it worked, would have a large impact on the
company's business.  My host supported me and gave me the help that I needed,
and the project turned out to be very successful.  The project was so
successful that the director of all of TV Ads at Google said that my project
was the most important work going on in the whole TV Ads team, and several
times he commended me for my work.  One of the main things I learned from the
experience was that while I enjoy writing software, I find much more
satisfication in solving interesting problems that actually improve people's
lives.  I was able to help the TV Ads team, which was struggling to make money,
greatly improve their business.

I returned from my internship with one year left as an undergraduate.  As that
year neared its end and the time came to apply to graduate schools, I found
that I had several research projects that needed just a little more time to be
finished.  I wanted to finish what I had started here at BYU.  I also wanted to
get into one of the nation's top universities for a PhD, and I thought that
having several publications, instead of just hopes of publishing, would give me
a much stronger application.  Further, I had taken several graduate-level
classes as an undergraduate that wouldn't transfer well to another university,
and I was really enjoying working with Dr. Seppi.  All of these pointed me
towards a masters degree at BYU, so I decided to stay, hoping to finish within
a year.

My first semester as a masters student during the summer of 2010 gave me my
first real teaching opportunity.  While I have tutored other students
throughout all of my college career, I had never done so in a formal setting.
That semester, however, I was a teaching assistant for two classes that Dr.
Seppi taught: Introduction to Articial Intelligence, and Bayesian Inference.  I
had always enjoyed helping people understand concepts, but I didn't realize how
much I would like the teaching aspect of being a professor until that
experience.  I found that figuring out how to explain complicated concepts
simply and seeing the light of understanding in the students' faces gave me
great satisfaction.

My experience so far as a masters student has served to strengthen my desire to
get a PhD.  I love learning and solving new problems.  I envision my future
career as one of research, likely in the domain of natural language processing,
though possibly also in computer vision or other areas of machine learning.  I
have not yet decided between research in industry or academia.  I enjoy
teaching, but often the resources available in industry allow for more cutting
edge research; the massive amounts of data and extensive computing resources at
Google allow them to tackle problems that are impossible on smaller scales.
Perhaps my career will take me first to industry, then back to academia as a
professor later in life.  But what I most want from my career, wherever I work,
is to find new problems whose solutions will better society, then solve them.
Areas such as machine translation and interpretation, or extracting knowledge
from large collections of text and making that knowledge accessible to people,
are intriguing problems that I would love to spend my life researching.


\end{document}
