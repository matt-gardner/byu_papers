\documentclass[onecolumn, 12pt]{article}
\usepackage{fullpage}
\usepackage{graphicx}

\begin{document}

Four of the items listed from my undergraduate career all came from a single
project focusing on how to effectively solve long-running optimization problems
with a large cluster of hundreds or thousands of machines.  The project started
when a physics professor approached my advisor seeking help solving a
crystallography problem.  Having just begun working with my advisor, I accepted
the problem mostly in an auxiliary role, writing some code and running
experiments for Dr. Campbell, the physics professor.  We used particle swarm
optimization, a bio-inspired optimization algorithm similar to evolutionary
algorithms, to try to fit to data his model of what was happening in a
particular kind of crystal.  After spending a lot of time seeing the output of
the algorithm, I also helped to interpret the results.  I discovered that we
had too many free parameters in our model and helped Dr. Campbell to make
progress in solving his problem.  

The crystallography problem we were solving required the evaluation of a
function that took several minutes to run.  As each particle in the algorithm
must evaluate the function at every iteration, we used several hundred
processors on BYU's supercomputer in our optimization of the function.  This
led us to ask some interesting questions about how we could best use those
processors to solve the problem.  We first experimented with the topology of
the algorithm, or how the particles in the swarm communicate with each other.
I did much of the experimentation, though I was directed by a graduate student
in the lab, who eventually wrote the paper that we published.  We saw some
improvements in the performance of the algorithm with our new topologies,
though we also noticed that increasing the number of particles with additional
processors had diminishing returns.

To solve this problem, my advisor proposed applying the idea of speculative
decomposition in the parallelization literature to particle swarm optimization
(PSO).  Instead of increasing the number of particles in the swarm, additional
processors could be used to evaluate possible next-iteration positions of each
particle in parallel.  I thought it was an interesting idea, and I made it my
honors thesis.  I took the idea and discovered how it could actually be done,
performing the evaluations needed and piecing the results back together from
messages sent between processors so that speculative evaulation could be done
in a distributed PSO environment with only a single round of message passing
per iteration.  After implementing the technique, I ran many experiments to
compare our methods to other means of parallelizing PSO.  Once I was satisfied
that my results were significant, I wrote, submitted, and successfully defended
my honors thesis.  I also modified the paper to fit in the aims and page
requirements of Parallel Problem Solving in Nature, a well-respected parallel
optimization conference, and submitted my thesis there, and it was accepted.

In implementing speculative evaluation, however, I noticed a number of ways the
method could potentially be improved.  Even though my honors thesis was
completed, I continued my investigation.  I found that a large number of the
speculative evaluations we performed were wasted, and that the algorithm could
be improved by replacing useless evaluations with better evaluations, such as
speculating several iterations ahead on paths that particles were likely to
take.  I significantly expanded the work of my honors thesis, preparing to
submit it to a journal on optimization.  After a rejection of a smaller version
from a conference, I got quite discouraged, wanting to give up and move on to
something new.  My advisor insisted I persevere, saying it was good
preparation for a PhD, and so I continued.  After running more experiments to
strengthen the paper, we submitted it to the journal Swarm Intelligence, where
it is currently under review.

While I have since move on to other interests, this project was my first
experience performing research and dealing with the submission process at
conferences.  All told, I worked on particle swarm optimization for about two
and a half years.  I learned a lot about what research is, the perseverance
required in getting a PhD, and how much I love exploring new topics and
extending the limits of what we can do.  This project set the stage for my
masters thesis and for my future work as a PhD student.  

\end{document}
