\documentclass[onecolumn, 12pt]{article}
\usepackage{fullpage}
\usepackage{graphicx}
\usepackage{cite}

\title{Using Natural Language Processing Techniques\\To Automatically Analyze
General Conference Talks}
\author{Matthew Gardner and Eric Ringger}
\date{}

\begin{document}
\maketitle

In natural language processing, topic identification is the problem of finding
the underlying semantic meaning in a set of documents.  More practically, it is
the problem of finding what topics are discussed in a corpus and in what
proportion each document contains each of those topics.  

Latent Dirichlet allocation (LDA) is a statistical model that is commonly used
to perform automatic topic identification.  It was first proposed in 2002 and
has been used with a number of different corpora since then~\cite{blei-jmlr03}.
In this paper we present the results of using LDA to analyze talks from the
past 100 years of LDS General Conferences.  We found our results to be
intriguing, correlating quite well in a number of instances with real-world
events.  The topics that were found by the algorithm included tithing, the
Sabbath day, Relief Society, the welfare program, and many more.  In looking at
the distribution of these topics over time we find some interesting patterns.
The tithing topic had its highest proportion in General Conference around the
year 1900, when President Lorenzo Snow made tithing a big focus.  The Relief
Society topic was almost completely absent until the 1970s, when women started
speaking in Conference.  The welfare topic had high points around the late
1930s and early 1940s (during the Great Depression) and in the 1970s, when a
separate welfare session was held for a few years.

This project has already been completed.  It is the first attempt of which we
are aware at using latent Dirichlet allocation or similar techniques to try to
extract meaning from General Conference talks.  We believe the results can be
used in a variety of ways, including for automatically discovering connections
between talks.

\bibliographystyle{plain}
\bibliography{/aml/home/mjg82/bib/nlp/lda/blei-jmlr03}

\end{document}
