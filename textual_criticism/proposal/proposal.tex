\documentclass[onecolumn, 11pt]{article}
\usepackage{fullpage}
\usepackage{graphicx}

\title{Automated Textual Criticism}
\author{Matthew Gardner}
\date{June 25, 2010}

\begin{document}
\maketitle

\section{Introduction}

Textual criticism can be loosely described as the analysis of a series of
manuscripts of a document (generally an ancient document) in an attempt to
reconstruct the original text produced by the author(s) of the document.
Because of the importance of religious documents and the proliferation of their
manuscripts, textual criticism has grown into a very large field, though its
use has not been confined solely to religious documents.

The process of textual criticism involves laboring over differing manuscripts
searching for discrepancies, recording them by hand and sifting through the
various manuscripts in order to come to some conclusion about the manuscripts'
origins.  Much of the work of textual criticism seems like it could be
automated, greatly reducing the workload on scholars and perhaps even producing
better results, by finding things that scholars miss.

\section{Objectives}

The main objective of this project is to explore the possibility of using
computers, especially statistical methods in natural language processing and
computational biology, to completely automate the process of textual criticism.
Were automated textual criticism possible, the resultant system would take as
input a collection of manuscripts and output a critical edition of the work
(with variant readings, probabilities that each of the readings was the
original reading, a history of changes to the manuscript, and so forth).  Such
a system is much too ambitious a project for a 598R project; I will seek only
to explore the feasibility of such a system, and to what extent people have
tried to build them already.

I will also try to build some portion of such a system, though what portion
that would be has yet to be determined, and will likely be influenced by my
initial research.  My efforts to build a system may also be hampered by a lack
of data to use; I do not yet know what the availability of manuscripts in
digital form is like.  So while building part of a system for automated textual
criticism is a desired objective, the main objective is the exploration of the
space to determine feasibility.

\section{Outline of proposed work}

The bulk of the work of this project will be reading, to familiarize myself
with the field and see what has been done before.  In particular, the reading
will include at least the following:

\begin{itemize}
  \item \emph{Misquoting Jesus}, by Bart Ehrman
  \item \emph{The Text of the New Testament}, by Bruce Metzger and Bart Ehrman
  \item \emph{How the New Testament Came to Be}, the proceedings of the Sperry
	Symposium in 2006
\end{itemize}

These books will give me an idea of the process of textual criticism and what
exactly it involves.  Understanding how modern textual critics do their work is
essential to automating the process.

I will also search what literature I can find about automating the process of
textual criticism.  Some has already been done, called cladistics (the
application of computational biology to textual criticism); this section of my
reading will explore to what extent such attempts have been made and how
successful they are.

Another important part of my work will be contacting individuals who work in
the field, to discuss the idea of automated textual criticism with them and to
see what access there is to relevant data sources.  If none of the manuscripts
are available digitally, automated methods are not possible.  I will talk at
least to BYU professors in the field, including Thomas Wayment, Don Parry,
David R. Seely, and Stephen Bay.

Finally, assuming that I find some amount of data to be available, I will try
to implement some part of an automated system for textual criticism.  Exactly
what part of the system that might be depends a lot on what I find in my
reading.  In the absence of any available data in digital form, I can at least
produce simulated data and test the results of my system on that.

I will also meet weekly with my faculty mentor, on Fridays at 11:00 AM, to
discuss my progress and get further direction.

\section{Deliverables}

The main deliverable of this project will be a paper summarizing my findings.
At the very least this will be a survey of the various parts of textual
criticism, what knowledge sources are required to accomplish each aspect of it,
my ideas about the feasibility of automating each of those parts, and what work
has already been done to automate them.  It is possible that such a survey
could be submitted for publication to a conference or journal, but I do not
know yet what conferences or journals are published in the field.

Assuming some part of the processes can be automated and I can find data to
use, a secondary deliverable is code that does some amount of automated textual
criticism.  Along with such code would go a paper describing the code, and if
it is deemed interesting enough, that paper could be submitted for publication
to some venue for textual criticism.  It seems to me, though, that producing a
publishable paper on top of all of the research and code writing that would be
involved might take a little longer than a term.  A reasonable deliverable to
me seems to be at least the beginnings of a paper that could be published.

\end{document}
